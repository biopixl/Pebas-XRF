% ==============================================================================
% Pebas-XRF: Manuscript for Plot Interpretation Review
% ==============================================================================
\documentclass[11pt,a4paper]{article}

% Packages
\usepackage[utf8]{inputenc}
\usepackage[T1]{fontenc}
\usepackage{graphicx}
\usepackage{booktabs}
\usepackage{longtable}
\usepackage{siunitx}
\usepackage[margin=2.5cm]{geometry}
\usepackage{natbib}
\usepackage{hyperref}
\usepackage{xcolor}
\usepackage{lineno}
\usepackage{setspace}

% Line numbers for review
\linenumbers
\onehalfspacing

% Custom commands
\newcommand{\ratio}[2]{#1/#2}
\newcommand{\TODO}[1]{\textcolor{red}{\textbf{[TODO: #1]}}}
\newcommand{\REVIEW}[1]{\textcolor{blue}{\textbf{[REVIEW: #1]}}}

% Title
\title{XRF Core Scanner Geochemistry of the Pebas Formation:\\
Paleoenvironmental Reconstruction from Tamshiyacu and Santa Corina Cores}

\author{Isaac Ayesu\textsuperscript{1}}

\date{\today \\ \small Draft for Internal Review}

\begin{document}

\maketitle

\begin{abstract}
This document presents preliminary XRF core scanner results from sediment cores collected from the Pebas Formation at Tamshiyacu (TAM) and Santa Corina (SC) localities in the Peruvian Amazon. We apply established geochemical proxies to interpret paleoenvironmental conditions including terrigenous input, redox state, chemical weathering intensity, and sediment provenance. Element ratios (Ca/Ti, Fe/Mn, K/Ti, Zr/Rb) are calculated following best practices for compositional data analysis, and principal component analysis (PCA) with centered log-ratio (CLR) transformation is used to identify multivariate geochemical signatures.
\end{abstract}

\tableofcontents
\newpage

% ==============================================================================
\section{Introduction}
% ==============================================================================

The Pebas Formation represents a critical archive of Miocene environmental conditions in western Amazonia. Understanding the geochemical signatures preserved in these sediments provides insights into:

\begin{itemize}
    \item Paleolake/wetland dynamics during the Miocene
    \item Andean uplift and its influence on sediment provenance
    \item Redox conditions and organic matter preservation
    \item Hydrological variability and terrigenous input
\end{itemize}

\subsection{Study Sites}

Two core localities were analyzed:

\begin{description}
    \item[Tamshiyacu (TAM)] \TODO{Add coordinates and stratigraphic context}
    \item[Santa Corina (SC)] \TODO{Add coordinates and stratigraphic context}
\end{description}

\subsection{Objectives}

\begin{enumerate}
    \item Characterize the major and trace element geochemistry of TAM and SC cores
    \item Apply XRF-based paleoenvironmental proxies with appropriate normalization
    \item Compare geochemical signatures between localities
    \item Interpret depositional environments using multivariate analysis
\end{enumerate}

% ==============================================================================
\section{Methods}
% ==============================================================================

\subsection{XRF Core Scanning}

Sediment cores were analyzed using an Itrax XRF core scanner (Cox Analytical Systems) with the following parameters:

\begin{table}[h]
\centering
\caption{Itrax XRF scanning parameters}
\label{tab:scan_params}
\begin{tabular}{ll}
\toprule
Parameter & Value \\
\midrule
X-ray tube & Molybdenum (Mo) \\
Voltage & \SI{30}{kV} \\
Current & \SI{55}{mA} \\
Exposure time & \SI{10}{s} \\
Step size & \SI{3}{mm} \\
\bottomrule
\end{tabular}
\end{table}

The Mo tube configuration provides optimal detection for elements from Al to Mo, with good sensitivity for major rock-forming elements (Si, Al, K, Ca, Ti, Fe) and key trace elements (Mn, Rb, Sr, Zr, Ba).

\subsection{Data Processing}

\subsubsection{Quality Control}

Measurements were filtered based on:
\begin{itemize}
    \item Mean Squared Error (MSE) of spectral fit: $\leq 10$
    \item Total counts per second (cps): $\geq 20,000$
    \item Sample surface distance: $< \SI{8}{mm}$ (detector proximity)
    \item Validity flag: Valid measurements only
\end{itemize}

Additionally, 29 exclusion zones were identified through expert visual review of optical core images aligned with XRF profiles. These zones corresponded to foam fills, core gaps, and stopped scans totaling 392 measurements (20\% of initial data). After quality control, 1,500 measurements from 25 sections were retained for analysis.

\begin{figure}[h]
\centering
\includegraphics[width=\textwidth]{figures/qc_qa_summary.png}
\caption{Quality control and quality assurance summary for retained XRF measurements. (A) Total X-ray counts distribution showing good signal quality (median $>$100,000 cps). (B) Spectral fit quality with all MSE values below QC threshold. (C) Element detection rates exceeding 98\% for all analyzed elements. (D) Data completeness by section. (E) Fe-Ti correlation validating detrital signal consistency (R$^2$ = 0.90 for SC, 0.60 for TAM). (F) Ca-Ti relationship showing carbonate vs. detrital end-member mixing.}
\label{fig:qc_qa}
\end{figure}

\subsubsection{Element Ratios}

Element ratios were calculated to minimize matrix effects and closed-sum artifacts inherent to XRF core scanner data \citep{Weltje2008, Croudace2015}. The following ratios were computed:

\begin{table}[h]
\centering
\caption{Element ratios and their paleoenvironmental interpretations}
\label{tab:ratios}
\begin{tabular}{lll}
\toprule
Ratio & Proxy For & Key References \\
\midrule
\ratio{Ca}{Ti} & Carbonate vs. terrigenous input & \citet{Haug2001, Davies2015} \\
\ratio{Fe}{Mn} & Redox conditions & \citet{Croudace2006, Rothwell2015} \\
\ratio{K}{Ti} & Chemical weathering intensity & \citet{Nesbitt1982} \\
\ratio{Rb}{Sr} & Silicate weathering & \citet{Jin2001} \\
\ratio{Zr}{Rb} & Grain size / energy & \citet{Dypvik2001} \\
\ratio{Si}{Ti} & Biogenic silica & \citet{Brown2007} \\
\bottomrule
\end{tabular}
\end{table}

\subsubsection{Compositional Data Analysis}

XRF element data are compositional (subject to closure constraints), which can produce spurious correlations in standard statistical analyses \citep{Aitchison1986}. Following best practices \citep{Weltje2015}, we applied centered log-ratio (CLR) transformation before principal component analysis:

\begin{equation}
\text{CLR}(x_i) = \ln\left(\frac{x_i}{g(\mathbf{x})}\right)
\end{equation}

where $g(\mathbf{x})$ is the geometric mean of all components.

% ==============================================================================
\section{Results}
% ==============================================================================

\subsection{Data Summary}

\begin{table}[h]
\centering
\caption{Summary of XRF measurements by core series after quality control}
\label{tab:data_summary}
\begin{tabular}{lrrrrrr}
\toprule
Core Series & Sections & Measurements & Ca/Ti & Fe/Mn & K/Ti \\
\midrule
TAM (Tamshiyacu) & 9 & 697 & 4.9 & 85 & 0.94 \\
SC (Santa Corina) & 16 & 803 & 3.0 & 50 & 0.90 \\
\midrule
Total & 25 & 1,500 & --- & --- & --- \\
\bottomrule
\end{tabular}
\end{table}

Key geochemical differences between cores (Table~\ref{tab:data_summary}):
\begin{itemize}
    \item \textbf{Ca/Ti ratio:} TAM shows higher median Ca/Ti (4.9) compared to SC (3.0), suggesting greater carbonate content or reduced terrigenous dilution at Tamshiyacu
    \item \textbf{Fe/Mn ratio:} TAM exhibits significantly higher Fe/Mn (85) than SC (50), with both values above the oxic/anoxic threshold of $\sim$40--60, indicating reducing bottom water conditions at both sites but more strongly reducing at TAM
    \item \textbf{K/Ti ratio:} Similar values between sites (0.90--0.94) suggest comparable weathering intensity in source areas
\end{itemize}

\subsection{Stratigraphic Profiles}

Stratigraphic profiles of major elements and key ratios were generated for all core sections. Representative composite profiles for each core series are shown in Figures~\ref{fig:strat_tam} and \ref{fig:strat_sc}.

\begin{figure}[h]
\centering
\includegraphics[width=\textwidth]{figures/stacked_TAM.png}
\caption{Composite stratigraphic column for Tamshiyacu (TAM) cores stacked in correct stratigraphic order (GROUP1 = top/youngest to GROUP3 = bottom/oldest). Total depth $\sim$3.6 m. Panels show Fe, Ca (element concentrations), Ca/Ti (carbonate proxy), Fe/Mn (redox proxy with oxic/anoxic threshold at 50), K/Ti (weathering intensity), and Zr/Rb (grain size). Note persistently high Fe/Mn values ($>$50) indicating reducing conditions throughout.}
\label{fig:strat_tam}
\end{figure}

\begin{figure}[h]
\centering
\includegraphics[width=\textwidth]{figures/stacked_SC.png}
\caption{Composite stratigraphic column for Santa Corina (SC) cores stacked in correct stratigraphic order (GROUP4 = top/youngest to GROUP7 = bottom/oldest). Total depth $\sim$4.0 m. SC shows lower overall Fe/Mn compared to TAM, though still predominantly above the oxic/anoxic threshold.}
\label{fig:strat_sc}
\end{figure}

\subsection{Element Ratio Distributions}

Element ratio distributions reveal distinct geochemical signatures between the two core localities (Figure~\ref{fig:ratio_dist}).

\begin{figure}[h]
\centering
\includegraphics[width=0.9\textwidth]{figures/ratio_distributions.png}
\caption{Kernel density distributions of key element ratios comparing Tamshiyacu (TAM, blue) and Santa Corina (SC, orange) cores. TAM shows higher Ca/Ti and Fe/Mn values, while K/Ti and Zr/Rb are similar between sites.}
\label{fig:ratio_dist}
\end{figure}

The ratio distributions indicate:
\begin{itemize}
    \item \textbf{Ca/Ti:} TAM distribution is shifted toward higher values, suggesting enhanced carbonate production or accumulation relative to clastic input
    \item \textbf{Fe/Mn:} Both sites show distributions above the oxic/anoxic threshold ($\sim$50), with TAM exhibiting a higher median indicating more strongly reducing conditions
    \item \textbf{K/Ti and Zr/Rb:} Overlapping distributions suggest similar source area weathering and grain size characteristics at both localities
\end{itemize}

\subsection{Principal Component Analysis}

PCA of CLR-transformed element data reveals the multivariate geochemical structure of the Pebas Formation sediments (Figure~\ref{fig:pca}).

\begin{figure}[h]
\centering
\includegraphics[width=\textwidth]{figures/pca_analysis.png}
\caption{Principal component analysis of CLR-transformed XRF element data. (A) Biplot showing sample scores colored by core series with element loading vectors. (B) Variance explained by each PC. (C) Element loadings on PC1 and PC2. (D) Hierarchical clustering of samples in PC space.}
\label{fig:pca}
\end{figure}

\subsubsection{PC1 Interpretation (Primary Geochemical Gradient)}

PC1 captures the dominant geochemical variability and likely represents a terrigenous input gradient:
\begin{itemize}
    \item \textbf{Positive loadings:} Fe, Ti, Al, K, Rb --- detrital/terrigenous elements associated with siliciclastic input
    \item \textbf{Negative loadings:} Ca, Sr --- carbonate-associated elements
    \item \textbf{Interpretation:} PC1 reflects the balance between terrigenous clastic input and authigenic/biogenic carbonate accumulation. Higher PC1 scores indicate greater terrigenous dominance.
\end{itemize}

\subsubsection{PC2 Interpretation (Secondary Gradient)}

PC2 captures secondary variability potentially related to redox or grain size:
\begin{itemize}
    \item \textbf{Positive loadings:} Si, Zr --- coarser fraction and heavy minerals
    \item \textbf{Negative loadings:} Mn, Fe --- redox-sensitive elements
    \item \textbf{Interpretation:} PC2 may reflect grain size variations (Zr enrichment in coarser fractions) or redox conditions affecting Mn mobility.
\end{itemize}

The PCA biplot shows partial separation between TAM and SC samples, with TAM generally plotting at lower PC1 values (more carbonate-rich) consistent with the higher Ca/Ti ratios observed in univariate analysis.

% ==============================================================================
\section{Proxy Interpretation Guide}
% ==============================================================================

\REVIEW{This section provides literature-based expectations for each proxy. Compare observed patterns against these benchmarks.}

\subsection{\ratio{Ca}{Ti} --- Carbonate vs. Terrigenous Input}

\subsubsection{Literature Basis}
\begin{itemize}
    \item Ti is a conservative, immobile element of purely terrigenous origin \citep{Calvert1996}
    \item Ca in lacustrine settings derives from: (1) authigenic carbonate precipitation, (2) biogenic carbonate (ostracods, mollusks), (3) detrital carbonate
    \item Higher \ratio{Ca}{Ti} indicates increased carbonate production or reduced clastic dilution
    \item In the Pebas Formation context: \TODO{What is expected based on known paleoecology?}
\end{itemize}

\subsubsection{Observed Pattern}
TAM cores show systematically higher Ca/Ti (median 4.9) compared to SC (median 3.0). Both sites show considerable variability with some intervals reaching Ca/Ti $>$10.

\subsubsection{Interpretation}
The elevated Ca/Ti at TAM suggests either: (1) greater biogenic carbonate production (ostracods, mollusks characteristic of Pebas lake systems), (2) increased authigenic carbonate precipitation, or (3) reduced terrigenous dilution. Given the endemic mollusk-rich fauna documented from the Pebas Formation, biogenic carbonate is the most likely driver of elevated Ca.

\subsection{\ratio{Fe}{Mn} --- Redox Conditions}

\subsubsection{Literature Basis}
\begin{itemize}
    \item Mn is more readily mobilized under reducing conditions than Fe \citep{Calvert1996}
    \item Under oxic bottom waters, Mn precipitates as oxides at the sediment-water interface
    \item Under anoxic/dysoxic conditions, Mn diffuses upward and escapes, leaving Fe enriched
    \item High \ratio{Fe}{Mn} $\rightarrow$ reducing (anoxic) conditions
    \item Low \ratio{Fe}{Mn} $\rightarrow$ oxidizing (oxic) conditions
    \item Typical oxic/anoxic threshold: \ratio{Fe}{Mn} $\approx$ 40--60 \citep{Rothwell2015}
\end{itemize}

\subsubsection{Observed Pattern}
Both cores show Fe/Mn values well above the oxic/anoxic threshold: TAM median = 85, SC median = 50. Values range from $\sim$30 to $>$500, with TAM consistently higher.

\subsubsection{Interpretation}
The elevated Fe/Mn ratios indicate predominantly reducing (dysoxic to anoxic) bottom water conditions at both Pebas localities. The higher values at TAM suggest more persistent or intense reducing conditions, potentially related to greater organic matter accumulation and oxygen consumption. This is consistent with reconstructions of the Pebas mega-wetland as a stratified water body with limited bottom water oxygenation.

\subsection{\ratio{K}{Ti} --- Chemical Weathering Intensity}

\subsubsection{Literature Basis}
\begin{itemize}
    \item K is mobile during chemical weathering; Ti is immobile
    \item Intense weathering depletes K relative to Ti (lower \ratio{K}{Ti})
    \item Higher \ratio{K}{Ti} indicates less weathered, more ``fresh'' detritus
    \item Related to Climate Index of Alteration (CIA) concept \citep{Nesbitt1982}
\end{itemize}

\subsubsection{Observed Pattern}
K/Ti values are similar between sites: TAM median = 0.94, SC median = 0.90. Both show relatively narrow distributions suggesting consistent source weathering characteristics.

\subsubsection{Interpretation}
The similar K/Ti ratios indicate that both localities received sediment from source areas with comparable weathering intensity. The moderate values suggest neither extreme weathering (which would deplete K) nor pristine unweathered material. This is consistent with derivation from a common Andean/cratonic source that underwent moderate tropical weathering before deposition.

\subsection{\ratio{Zr}{Rb} --- Grain Size Proxy}

\subsubsection{Literature Basis}
\begin{itemize}
    \item Zr concentrates in heavy minerals (zircon) in coarser fractions
    \item Rb associates with clay minerals (illite, muscovite) in finer fractions
    \item Higher \ratio{Zr}{Rb} indicates coarser sediment, higher depositional energy
    \item Lower \ratio{Zr}{Rb} indicates finer sediment, lower energy (lacustrine, distal)
    \item Validated against laser grain size in multiple studies \citep{Dypvik2001, Kylander2011}
\end{itemize}

\subsubsection{Observed Pattern}
Zr/Rb shows overlapping distributions between TAM and SC, with median values near 0.7--0.8. Some stratigraphic variability is evident but without systematic site differences.

\subsubsection{Interpretation}
The relatively low Zr/Rb values indicate fine-grained sedimentation characteristic of low-energy lacustrine or distal floodplain environments. The lack of systematic difference between sites suggests similar hydrodynamic regimes and sediment sorting processes. Occasional higher Zr/Rb intervals may record increased current energy or coarser sediment input events.

\subsection{\ratio{Rb}{Sr} --- Silicate Weathering}

\subsubsection{Literature Basis}
\begin{itemize}
    \item Sr is preferentially leached during silicate weathering; Rb is retained in clays
    \item Higher \ratio{Rb}{Sr} indicates more intense chemical weathering of source rocks
    \item Can also reflect provenance differences (felsic vs mafic sources)
    \item Used extensively in loess-paleosol studies \citep{Jin2001}
\end{itemize}

\subsubsection{Observed Pattern}
Rb/Sr shows modest variability with TAM generally exhibiting higher values than SC, reflecting the inverse relationship with Ca (since Sr associates with carbonate).

% ==============================================================================
\section{Discussion}
% ==============================================================================

\subsection{Comparison of TAM and SC Geochemistry}

The XRF data reveal both similarities and differences between the Tamshiyacu and Santa Corina cores:

\begin{enumerate}
    \item \textbf{Similar features:} Both sites show (a) reducing bottom water conditions (elevated Fe/Mn), (b) fine-grained sedimentation (low Zr/Rb), and (c) comparable source weathering (similar K/Ti). This suggests deposition within the same broad lacustrine/wetland system.

    \item \textbf{Key differences:} TAM exhibits higher Ca/Ti and Fe/Mn than SC, indicating (a) greater carbonate accumulation and (b) more strongly reducing conditions at the Tamshiyacu locality.

    \item \textbf{PCA interpretation:} The partial separation of TAM and SC samples along PC1 confirms that carbonate content is the primary discriminating factor between sites, while overlap on PC2 indicates similar grain size and provenance.
\end{enumerate}

\subsection{Paleoenvironmental Reconstruction}

\subsubsection{Depositional Environment}
The geochemical data support deposition in a low-energy lacustrine or marginal wetland environment:
\begin{itemize}
    \item Low Zr/Rb ratios indicate fine-grained, clay-rich sedimentation
    \item Elevated carbonate (Ca/Ti) is consistent with endemic mollusk-rich Pebas fauna
    \item Lack of marine indicators (e.g., Br enrichment) supports freshwater to brackish conditions
\end{itemize}

\subsubsection{Redox Conditions}
Both cores record predominantly reducing conditions:
\begin{itemize}
    \item Fe/Mn ratios consistently above the oxic/anoxic threshold (50--85)
    \item More intense reducing conditions at TAM may reflect greater water depth or organic loading
    \item Consistent with reconstructions of stratified Pebas mega-wetland
\end{itemize}

\subsubsection{Terrigenous Input and Weathering}
Sediment provenance appears consistent between sites:
\begin{itemize}
    \item Similar K/Ti suggests common source area weathering
    \item Strong Fe-Ti correlation (R$^2$ = 0.60--0.90) confirms coherent detrital signal
    \item Variations in terrigenous input modulate carbonate dilution
\end{itemize}

\subsection{Comparison with Published Pebas Studies}

These geochemical signatures are broadly consistent with previous reconstructions of the Pebas Formation as a long-lived, shallow lacustrine to marginal marine system \citep{Wesselingh2002, Hoorn2010}. The reducing conditions inferred from Fe/Mn support organic-rich, stratified water bodies. The carbonate component reflects the endemic mollusk fauna characteristic of Pebas deposits.

% ==============================================================================
\section{Conclusions}
% ==============================================================================

XRF core scanner analysis of Pebas Formation sediments from Tamshiyacu (TAM) and Santa Corina (SC) reveals:

\begin{enumerate}
    \item \textbf{Data quality:} After rigorous QC filtering and expert review of optical images, 1,500 measurements from 25 sections provide reliable geochemical data with $>$98\% element detection rates and consistent spectral fits.

    \item \textbf{Redox conditions:} Elevated Fe/Mn ratios (50--85) at both sites indicate predominantly reducing bottom water conditions.

    \item \textbf{Site differences:} TAM shows higher carbonate content (Ca/Ti = 4.9 vs 3.0) and more strongly reducing conditions (Fe/Mn = 85 vs 50) than SC, suggesting spatial heterogeneity within the Pebas depositional system.

    \item \textbf{Depositional environment:} Low Zr/Rb ratios and fine-grained sedimentation support low-energy lacustrine or marginal wetland deposition, consistent with the Pebas mega-wetland reconstruction.

    \item \textbf{Provenance:} Similar K/Ti ratios and strong Fe-Ti correlations indicate consistent terrigenous source characteristics between sites.
\end{enumerate}

These results validate XRF core scanning as an effective tool for paleoenvironmental reconstruction in Amazonian Miocene sediments and provide baseline geochemical characterization for ongoing stratigraphic and paleontological studies of the Pebas Formation.

% ==============================================================================
% References
% ==============================================================================

\bibliographystyle{apalike}
\bibliography{references}

\end{document}
