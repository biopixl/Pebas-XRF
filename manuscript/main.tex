% ==============================================================================
% Pebas Formation XRF Core Scanner Stratigraphy
% Detailed Geochemical Documentation of Tamshiyacu and Santa Corina Cores
% ==============================================================================
\documentclass[11pt,a4paper]{article}

% Packages
\usepackage[utf8]{inputenc}
\usepackage[T1]{fontenc}
\usepackage{graphicx}
\usepackage{booktabs}
\usepackage{longtable}
\usepackage{siunitx}
\usepackage[margin=2.5cm]{geometry}
\usepackage{natbib}
\usepackage{hyperref}
\usepackage{xcolor}
\usepackage{lineno}
\usepackage{setspace}
\usepackage{rotating}
\usepackage{pdflscape}
\usepackage{multirow}
\usepackage{caption}
\usepackage{subcaption}
\usepackage{adjustbox}  % For image brightness/contrast enhancement

% Image enhancement command for publication quality
% Usage: \brightimage[width=\textwidth,brightness=1.2]{path/to/image.png}
\newcommand{\brightimage}[2][]{%
  \adjustbox{cfbox=none,raise=-0.5ex}{%
    \includegraphics[#1]{#2}%
  }%
}

% Line numbers for review
\linenumbers
\onehalfspacing

% Custom commands
\newcommand{\ratio}[2]{#1/#2}
\newcommand{\CaTi}{Ca/Ti}  % Deprecated - retained for backwards compatibility
\newcommand{\FeMn}{Fe/Mn}
\newcommand{\KTi}{K/Ti}
\newcommand{\ZrRb}{Zr/Rb}
\newcommand{\RbSr}{Rb/Sr}

% Title
\title{Itrax XRF Core Scanner Stratigraphy of the Pebas Formation, Western Amazonia:\\
Detailed Geochemical Documentation of Tamshiyacu and Santa Corina Sediment Cores}

\author{Isaac Ayesu\textsuperscript{1}}

\date{\today}

\begin{document}

\maketitle

\begin{abstract}
This report presents a comprehensive stratigraphic and geochemical documentation of sediment cores from the Miocene Pebas Formation at two localities in the Peruvian Amazon: Tamshiyacu (TAM) and Santa Corina (SC). Seven core groups comprising 27 sections were analyzed using Itrax XRF core scanning at 3~mm resolution, yielding 1,979 calibrated measurements after quality control. We document the stratigraphic position, geochemical characteristics, and paleoenvironmental interpretation for each core section using a statistically-validated proxy suite: Ca for authigenic/biogenic carbonate, Ti for terrigenous flux, \FeMn{} for redox conditions, and \ZrRb{} for grain size. Empirical redundancy analysis demonstrates that the commonly-used Ca/Ti ratio is statistically redundant with raw Ca ($r$ = 0.89) in this dataset, while \FeMn{} and \ZrRb{} provide genuinely independent information. The data reveal systematic differences between sites---TAM shows higher carbonate content and more reducing conditions than SC---and depth-dependent variations in depositional environment. This detailed core-by-core documentation provides a reference framework for ongoing paleontological and sedimentological studies of the Pebas mega-wetland system.
\end{abstract}

\tableofcontents
\newpage

% ==============================================================================
\section{Introduction}
% ==============================================================================

\subsection{Geological Setting}

The Pebas Formation (middle to late Miocene, ca.\ 23--10 Ma) represents one of the largest freshwater to marginally brackish lake-wetland systems in Earth history, covering an estimated 1 million km$^2$ of western Amazonia \citep{Wesselingh2002, Hoorn2010}. The formation is characterized by cyclic sequences of lignites, mudstones, and shell beds deposited in shallow lacustrine, fluvial, and marginal marine environments associated with the Pebas mega-wetland system.

\subsection{Study Localities}

Two outcrop localities in the Peruvian Amazon were selected for detailed XRF core scanning:

\begin{description}
    \item[Tamshiyacu (TAM):] Located along the Amazon River approximately 30~km downstream of Iquitos (coordinates: 4.0017\textdegree S, 73.1567\textdegree W). Three core groups (GROUP1--3) spanning approximately 3.9~m of composite stratigraphy.

    \item[Santa Corina (SC):] Located on the Itaya River south of Iquitos (coordinates: 4.1833\textdegree S, 73.4333\textdegree W). Four core groups (GROUP4--7) spanning approximately 4.0~m of composite stratigraphy.
\end{description}

\subsection{Objectives}

This report provides:
\begin{enumerate}
    \item Complete inventory of all analyzed core sections with stratigraphic positions
    \item Section-by-section geochemical characterization using calibrated XRF data
    \item Integrated optical core images with geochemical profiles
    \item Facies classification based on element ratio thresholds
    \item Paleoenvironmental interpretation for each core group
\end{enumerate}

% ==============================================================================
\section{Materials and Methods}
% ==============================================================================

\subsection{Core Collection and Sampling}

Sediment cores were collected using a manual percussion coring system with 50~mm diameter aluminum core tubes. Cores were split longitudinally, described, and photographed before XRF analysis. Core sections were designated with site prefix (TAM or SC), core number, and section letter (A, B, C, etc.). Some sections required re-scanning (designated ``RUN2'').

\subsection{Itrax XRF Core Scanning}

All cores were analyzed using an Itrax XRF core scanner (Cox Analytical Systems, Sweden) at [Institution]. Scanning parameters are summarized in Table~\ref{tab:scan_params}.

\begin{table}[h]
\centering
\caption{Itrax XRF core scanner operating parameters}
\label{tab:scan_params}
\begin{tabular}{ll}
\toprule
Parameter & Value \\
\midrule
X-ray tube & Molybdenum (Mo) \\
Voltage & \SI{30}{kV} \\
Current & \SI{55}{mA} \\
Exposure time & \SI{10}{s} per measurement \\
Step size & \SI{3}{mm} \\
Optical image resolution & \SI{0.2}{mm} \\
Detector & Silicon drift detector (SDD) \\
\bottomrule
\end{tabular}
\end{table}

\subsection{CT Core Scanning}

Select core sections were also analyzed using computed tomography (CT) scanning to provide complementary density information. CT scans were acquired using a Toshiba Aquilion clinical CT scanner with a bone reconstruction kernel (FC30), optimized for mineral differentiation. Scanning parameters included:

\begin{itemize}
    \item Slice thickness: \SI{2.0}{mm}
    \item Pixel spacing: \SI{0.47}--\SI{0.62}{mm}
    \item Image matrix: $512 \times 512$ pixels
    \item Two coronal series per sample: TOP and BOTTOM reference scans
\end{itemize}

The CT data provides X-ray attenuation (density) information that complements XRF elemental data. CT intensity correlates with bulk density and is sensitive to carbonate content, shell concentrations, and porosity variations. DICOM metadata (SliceLocation values) enables depth-matching with XRF position data for integrated analysis.

\subsection{Data Processing and Calibration}

Raw XRF spectra were processed using the Q-spec software with post-processing calibration (Results.txt files). Element concentrations are reported as calibrated peak areas proportional to concentration. Quality control criteria included:

\begin{itemize}
    \item Mean Squared Error (MSE) of spectral fit: $\leq 10$
    \item Total counts per second (cps): $\geq 20,000$
    \item Sample surface distance: $< \SI{8}{mm}$
    \item Visual inspection of optical images to identify foam, gaps, and artifacts
\end{itemize}

\subsection{Element Ratio Proxies}

Following established protocols for XRF core scanner paleoenvironmental reconstruction \citep{Croudace2015, Rothwell2015}, we evaluated multiple elemental proxies for their utility in the Pebas Formation context. A key finding of this study is that the commonly-used Ca/Ti ratio is statistically redundant with raw Ca counts ($r$ = 0.89), because Ca exhibits substantially higher variance (CV = 81\%) than Ti (CV = 36\%). We therefore adopt raw Ca as the primary carbonate indicator. Table~\ref{tab:proxies} summarizes the recommended proxy suite.

\begin{table}[h]
\centering
\caption{Recommended geochemical proxies for Pebas Formation analysis}
\label{tab:proxies}
\begin{tabular}{llll}
\toprule
Proxy & Type & Interpretation & Notes \\
\midrule
\multicolumn{4}{l}{\textit{Tier 1: Primary proxies}} \\
Ca & Element & Authigenic/biogenic carbonate & High variance (CV = 81\%) \\
Ti & Element & Terrigenous detrital flux & Conservative tracer \\
\FeMn{} & Ratio & Redox conditions & $>$50: reducing; $<$50: oxic \\
\ZrRb{} & Ratio & Grain size/energy & Higher = coarser sediment \\
\midrule
\multicolumn{4}{l}{\textit{Tier 2: Supporting proxies}} \\
Fe & Element & Lateritic input & Weathered Fe-oxides \\
Sr & Element & Carbonate mineralogy & Aragonite $\gg$ calcite \\
\KTi{} & Ratio & Provenance/weathering & K depletion indicator \\
\bottomrule
\end{tabular}
\end{table}

A detailed evaluation of proxy selection methodology, including redundancy analysis, detection quality assessment, and literature support for proxy mechanisms, is presented in Section~\ref{sec:proxy_evaluation}.

\subsection{Facies Classification}

A four-fold geochemical facies classification was applied based on Ca counts (in kcps):

\begin{description}
    \item[Shell-rich (Ca $>$ 100 kcps):] Biogenic carbonate-dominated intervals, typically containing mollusk shell accumulations
    \item[Carbonate (Ca 50--100 kcps):] Carbonate-enriched mudstones with mixed biogenic and authigenic carbonate
    \item[Mixed (Ca 20--50 kcps):] Transitional facies with balanced carbonate and terrigenous input
    \item[Clastic (Ca $<$ 20 kcps):] Terrigenous-dominated siliciclastic mudstones
\end{description}

Note: For compatibility with previous literature using Ca/Ti ratios, we report both metrics in summary tables. The Ca/Ti thresholds ($>$10: shell-rich, 5--10: carbonate, 2--5: mixed, $<$2: clastic) correspond approximately to the Ca thresholds above.

% ==============================================================================
% Include detailed proxy evaluation methodology
% ==============================================================================
% ==============================================================================
% PROXY EVALUATION SECTION
% XRF Geochemical Proxies for Pebas Formation Paleoenvironmental Reconstruction
% ==============================================================================

\section{Geochemical Proxy Evaluation}
\label{sec:proxy_evaluation}

The application of XRF-derived elemental proxies to Miocene lacustrine sediments requires careful evaluation of both analytical limitations and paleoenvironmental relevance. We present a systematic assessment of proxy utility for the Pebas Formation based on three criteria: (1) element detection quality with the Itrax Mo tube, (2) statistical independence from raw element counts, and (3) mechanistic relevance to tropical freshwater mega-wetland settings.

\subsection{Analytical Considerations}
\label{subsec:analytical}

XRF core scanning with a molybdenum (Mo) X-ray tube provides optimal detection for elements with atomic numbers greater than vanadium (Z $>$ 23), while lighter elements (Al, Si, P, S) suffer from reduced sensitivity \citep{Croudace2006}. The Mo tube is relatively inefficient in detecting light elements like Al or Si; the Cr tube is more effective for elements lighter than Ti, while the Mo tube excels for heavier elements \citep{Gebregiorgis2020}. Response rates for elements such as K, Ti, Si, and Al range from approximately 40,000 to 10,000 cps in decreasing order, with heavier elements (Rb, Sr, Zr) showing excellent detection due to their higher atomic numbers \citep{Croudace2006}. Table~\ref{tab:detection} summarizes detection quality for elements relevant to this study.

\begin{table}[htbp]
\centering
\caption{Element detection quality with Itrax Mo tube at 60 kV, 30 mA}
\label{tab:detection}
\begin{tabular}{llll}
\toprule
\textbf{Element} & \textbf{Z} & \textbf{Detection} & \textbf{Notes} \\
\midrule
Fe & 26 & Excellent & Primary Mo tube strength \\
Ti & 22 & Very Good & Stable conservative element \\
Ca & 20 & Very Good & High response rate \\
Mn & 25 & Good & Adequate for redox proxies \\
Rb & 37 & Excellent & High-Z advantage \\
Sr & 38 & Excellent & High-Z advantage \\
Zr & 40 & Excellent & High-Z advantage \\
K & 19 & Good & Above sensitivity threshold \\
\midrule
Al & 13 & Marginal & Reliable only $>$22,000 ppm; Cr tube preferred \\
Si & 14 & Marginal & Cr tube preferred; matrix effects significant \\
\bottomrule
\end{tabular}
\end{table}

The elements detected depend on concentrations present in the sample, the sample matrix, the dwell time at each sample point, and the selection of X-ray tube \citep{Croudace2015}. A 15-second dwell time generally yields acceptable results for most common elements, though longer exposure times (20--30 s) improve detection of light elements and trace elements \citep{Sakamoto2006}.

\subsection{Proxy Redundancy Analysis}
\label{subsec:redundancy}

Elemental ratios are commonly employed to normalize for matrix effects and sediment dilution \citep{Weltje2008}. However, when the normalizing element (denominator) shows limited variance relative to the numerator, the resulting ratio becomes statistically redundant with the raw element count. We evaluated this by computing Pearson correlations between each ratio and its numerator element (Figure~\ref{fig:redundancy}).

\begin{figure}[htbp]
\centering
\includegraphics[width=\textwidth]{figures/proxy_evaluation/fig_S1_redundancy_analysis.png}
\caption{Proxy redundancy analysis. (A) Ca/Ti is highly correlated with raw Ca ($r$ = 0.89), indicating Ti normalization adds minimal information. (B--C) Fe/Mn and Zr/Rb show lower correlations with their numerators, confirming these ratios provide independent signals. (D) Summary of redundancy assessment; ratios exceeding the threshold ($|r| > 0.85$) are considered redundant.}
\label{fig:redundancy}
\end{figure}

The Ca/Ti ratio, widely used as a carbonate vs.\ detrital indicator \citep{Rothwell2015}, proved statistically redundant in this dataset ($r$ = 0.89 with raw Ca). This results from Ca exhibiting substantially higher coefficient of variation (CV = 81\%) than Ti (CV = 36\%), such that the ratio is dominated by Ca variability. In contrast, Fe/Mn ($r$ = 0.27) and Zr/Rb ($r$ = 0.60) ratios provide genuinely independent information beyond their constituent elements (Table~\ref{tab:redundancy}).

\begin{table}[htbp]
\centering
\caption{Proxy redundancy evaluation based on correlation with numerator element}
\label{tab:redundancy}
\begin{tabular}{llll}
\toprule
\textbf{Ratio} & \textbf{Correlation ($r$)} & \textbf{Status} & \textbf{Recommendation} \\
\midrule
Ca/Ti & 0.89 & Redundant & Use raw Ca \\
Ba/Ti & 0.74 & Borderline & Consider raw Ba \\
Zr/Rb & 0.60 & Useful & Retain \\
Fe/Ti & 0.57 & Useful & Retain \\
K/Ti & 0.44 & Useful & Retain \\
Fe/Mn & 0.27 & Useful & Retain \\
Rb/Sr & $-$0.11 & Useful & Retain \\
\bottomrule
\end{tabular}
\end{table}

\subsection{Recommended Proxy Suite}
\label{subsec:proxies}

Based on the above evaluation, we adopt a tiered proxy framework optimized for the Pebas Formation paleoenvironmental context (Table~\ref{tab:proxies}). The following subsections provide detailed justification for each proxy based on geochemical mechanisms and supporting literature.

\begin{table}[htbp]
\centering
\caption{Recommended proxy suite for Pebas Formation XRF analysis}
\label{tab:proxies}
\begin{tabular}{p{2cm}p{2cm}p{5cm}p{4cm}}
\toprule
\textbf{Proxy} & \textbf{Type} & \textbf{Interpretation} & \textbf{Mechanism} \\
\midrule
\multicolumn{4}{l}{\textit{Tier 1: Primary proxies (main figures)}} \\
Ca & Element & Carbonate/authigenic signal & Authigenic precipitation; biogenic shell material \\
Ti & Element & Terrigenous detrital flux & Conservative heavy minerals (ilmenite, rutile, titanite) \\
Fe/Mn & Ratio & Redox conditions & Differential Mn$^{2+}$ mobility under anoxia \\
Zr/Rb & Ratio & Grain size/energy & Zr in zircon (coarse); Rb in clays (fine) \\
\midrule
\multicolumn{4}{l}{\textit{Tier 2: Supporting proxies (supplementary)}} \\
Fe & Element & Lateritic input & Weathered Fe-oxides/hydroxides from catchment \\
Sr & Element & Carbonate mineralogy & Sr partitioning: aragonite $\gg$ calcite \\
K/Ti & Ratio & Provenance/weathering & K depletion in intense weathering; illite indicator \\
Rb/Sr & Ratio & Carbonate influence & Inverse indicator of Sr enrichment in carbonates \\
\bottomrule
\end{tabular}
\end{table}

\subsubsection{Calcium (Ca): Authigenic and Biogenic Carbonate}
\label{subsubsec:ca}

\paragraph{Geochemical Mechanism.}
Calcium in lacustrine sediments derives primarily from authigenic carbonate precipitation and biogenic shell material (molluscs, ostracods). Authigenic carbonate minerals (calcite, aragonite) precipitate when calcium reacts with dissolved inorganic carbon (DIC) under favorable supersaturation conditions \citep{Shapley2005}. In groundwater-influenced lakes, the authigenic carbonate flux (ACF) can become limited by water column cation availability and thereby coupled to groundwater inflow rates and aquifer recharge, with enhanced carbonate production corresponding to wet climatic periods \citep{Shapley2005}.

Lacustrine authigenic carbonates typically precipitate during summer when carbonate saturation peaks and solubility is depressed in the epilimnion \citep{Li2020}. The carbon and oxygen stable isotope ratios ($\delta^{13}$C and $\delta^{18}$O) of these carbonates serve as established paleoclimate proxies \citep{Falster2018}, and XRF-derived Ca concentrations provide a rapid, non-destructive means to identify carbonate-rich intervals for targeted isotopic analysis.

\paragraph{Application to Pebas Formation.}
The Pebas mega-wetland system hosted abundant endemic molluscs and ostracods whose aragonitic shells are exceptionally well-preserved \citep{Wesselingh2002, Vonhof2003}. Strontium, oxygen, and carbon isotope analyses of these shells confirm predominantly freshwater conditions with Andean-sourced drainage \citep{Vonhof1998, Vonhof2003}. Elevated Ca values in XRF profiles thus indicate either (1) authigenic carbonate precipitation during drier, more evaporative conditions with reduced clastic dilution, or (2) accumulation of biogenic shell material reflecting favorable conditions for mollusc and ostracod populations.

\paragraph{Detection Quality and Validation.}
Calcium detection with the Mo tube is very good (Table~\ref{tab:detection}), with high count rates providing reliable quantification. Previous studies demonstrate statistically significant correlations between XRF Ca/Ti ratios and conventionally measured carbonate concentrations \citep{Cuven2010}. The strong positive correlation between Ca and Sr in our dataset ($r$ = 0.62; Figure~\ref{fig:crossplots}D) supports a common carbonate source, consistent with Sr incorporation into aragonite shells \citep{Gabitov2006}.

\subsubsection{Titanium (Ti): Conservative Detrital Flux Indicator}
\label{subsubsec:ti}

\paragraph{Geochemical Mechanism.}
Titanium is recognized as lithophile, incompatible, and fluid-immobile, with isotopic composition insusceptible to water-rock interactions such as weathering and diagenesis \citep{Aarons2020, Deng2019}. Additionally, Ti is biologically inactive and unaffected by the biogeochemical reorganizations common in surface environments \citep{Aarons2020}. These properties establish Ti as a conservative tracer of siliciclastic detrital input.

In sediments, Ti is hosted primarily in resistant heavy minerals including ilmenite (FeTiO$_3$), rutile (TiO$_2$), titanite (CaTiSiO$_5$), and titaniferous magnetite \citep{Young2014}. These minerals, released from parent rocks by weathering, accumulate by density sorting in fluvial and lacustrine settings. While limited Ti isotope fractionation can occur under extreme chemical weathering \citep{Deng2019}, crustal protolith composition and sorting during transport exert stronger control on Ti distribution than weathering intensity \citep{Aarons2020}.

\paragraph{Application to Pebas Formation.}
The Pebas system received terrigenous sediment from both Andean and cratonic (shield) sources \citep{Hoorn2010, Latrubesse2010}. Ti concentrations track the relative contribution of siliciclastic material, with higher values indicating enhanced terrigenous flux during wetter periods with increased runoff. The anti-correlation between Ca and Ti (Figure~\ref{fig:crossplots}A) reflects end-member mixing between carbonate-dominated and siliciclastic-dominated sedimentation regimes.

\paragraph{Detection Quality and Validation.}
Ti and Al are the most commonly used reference elements for normalization in lake sediment studies \citep{Boyle2001}. Ti shows excellent detection with the Mo tube (response rates $\sim$40,000 cps) and remains stable once deposited, unaffected by diagenetic processes \citep{Boyle2001}. Strong positive correlations with other detrital elements (Ti--Rb: $r$ = 0.81; Ti--K: $r$ = 0.77; Ti--Fe: $r$ = 0.72) confirm Ti tracks coherent terrigenous sediment delivery.

\subsubsection{Fe/Mn Ratio: Redox Proxy}
\label{subsubsec:femn}

\paragraph{Geochemical Mechanism.}
Iron and manganese are commonly used proxies for tracing past bottom-water oxygenation because they rapidly change oxidation state and solubility with changing redox conditions \citep{Davison1993, Naeher2013}. The Fe/Mn ratio exploits the difference in redox potential between the Mn$^{3+}$/Mn$^{2+}$ half reaction (E$^\circ$ = +1.50 V) and the Fe$^{3+}$/Fe$^{2+}$ half reaction (E$^\circ$ = +0.67 V) \citep{Herndon2018}.

Under reducing (anoxic) conditions at the sediment-water interface, Mn$^{2+}$ is preferentially mobilized relative to Fe$^{2+}$ due to Mn's higher solubility and slower oxidation kinetics \citep{Davison1993}. Reduced Mn diffuses upward and may escape to the water column, while Fe is retained as sulfides or oxyhydroxides. This differential mobility leads to elevated Fe/Mn ratios in sediments deposited under anoxic bottom waters \citep{Calvert1996}.

\paragraph{Limitations and Caveats.}
The interpretation of Fe/Mn as a redox proxy requires careful consideration of several complicating factors \citep{Naeher2013, Scholtysik2021}:

\begin{enumerate}
\item \textbf{Detrital input:} Variable detrital Fe from lateritic catchment soils can obscure the redox signal. Periods with higher detrital Fe input reduce the applicability of the ratio \citep{Scholtysik2021}.

\item \textbf{Diagenetic trapping:} Under permanent anoxia, intensified early diagenetic processes trap Mn in sediments as carbonates (rhodochrosite), crystalline oxides, and humic-bound forms \citep{Scholtysik2021}. This Mn retention inverts the expected relationship.

\item \textbf{Geochemical focusing:} Redistributive transport of redox-sensitive metals along depth gradients concentrates Fe and Mn at local depressions, complicating interpretation \citep{Scholtysik2022}.

\item \textbf{Lake-specific calibration:} The single use of XRF-Mn/Fe is often not conclusive for inferring past redox conditions; application requires accounting for individual lake characteristics \citep{Scholtysik2021}.
\end{enumerate}

\paragraph{Application to Pebas Formation.}
Despite these limitations, Fe/Mn provides useful first-order redox discrimination when interpreted conservatively. Our data reveal systematically higher Fe/Mn in TAM cores (median = 93.4) compared to SC cores (median = 50.7), suggesting more reducing conditions in the TAM depositional setting (Figure~\ref{fig:crossplots}B). This difference is maintained across the full range of Fe concentrations, indicating a genuine environmental signal rather than a compositional artifact.

Under holomixis (complete lake mixing), the XRF-Mn/Fe ratio successfully reflects lake redox conditions \citep{Naeher2013}. We restrict Fe/Mn interpretation to identification of major redox transitions rather than subtle variations, acknowledging that detrital Fe from the lateritic Amazonian catchment introduces uncertainty.

\paragraph{Detection Quality.}
Both Fe (Z = 26) and Mn (Z = 25) show good to excellent detection with the Mo tube, and the ratio adds substantial independent information beyond raw Fe counts ($r$ = 0.27 with Fe), confirming its utility as a distinct proxy.

\subsubsection{Zr/Rb Ratio: Grain Size Proxy}
\label{subsubsec:zrrb}

\paragraph{Geochemical Mechanism.}
The Zr/Rb ratio serves as a grain size proxy based on the contrasting mineralogical hosts of these elements \citep{Dypvik2001, Chen2006}. Rubidium substitutes for potassium in clay minerals (illite, muscovite), micas, and K-feldspars, concentrating in fine-grained fractions \citep{Wu2020}. Zirconium occurs primarily in zircon (ZrSiO$_4$), a resistant heavy mineral that concentrates in coarser sediment fractions due to its high density \citep{Dypvik2001}.

Grain-size separation experiments confirm that Zr and Rb concentrate in different grain-size fractions, with the ln(Zr/Rb) ratio showing consistent relationships with sortable silt percent (SS\%) and sortable silt mean (SSM) grain-size parameters \citep{Wu2020}. Universal gradients exist in plots of ln(Zr/Rb) versus SS\% (34.1) and versus SSM (12.7), enabling semi-quantitative grain size estimation from XRF data \citep{Wu2020}.

\paragraph{Advantages Over Alternative Proxies.}
While Si/Al ratios also correlate with grain size, interpretations can be complicated by biogenic silica from diatoms, sponge spicules, and radiolarians \citep{Wu2020}. A major benefit of Zr/Rb is its insensitivity to biogenic inputs and to coarse quartz-rich material that would perturb Si/Al ratios. Additionally, the ln(Zr/Rb) ratio is insensitive to Mn-oxides/hydroxides and Fe-oxides/hydroxides variations \citep{Wu2020}.

\paragraph{Application to Pebas Formation.}
Higher Zr/Rb values indicate coarser sediment and higher depositional energy, potentially tracking fluvial channel proximity, flood events, or lake level changes affecting shoreline position. The ratio adds substantial information beyond raw Zr counts ($r$ = 0.60), and both elements show excellent detection with the Mo tube (Rb: Z = 37; Zr: Z = 40).

\paragraph{Limitations.}
Provenance effects may influence Zr/Rb if sediment sources (Andean vs.\ cratonic) differ in Zr/Rb signatures. Additionally, heavy mineral concentration (placer effects) can elevate Zr independently of bulk grain size. Log transformation [ln(Zr/Rb)] improves linearity with measured grain size \citep{Wu2020}.

\subsubsection{Strontium (Sr): Carbonate Mineralogy and Salinity}
\label{subsubsec:sr}

\paragraph{Geochemical Mechanism.}
Strontium partitions strongly between carbonate polymorphs, with aragonite incorporating substantially more Sr than calcite. The distribution coefficient ($K_{Sr}$) for aragonite is approximately 1.1 (essentially no discrimination between Ca and Sr), while calcite has $K_{Sr}$ $\approx$ 0.08 \citep{Gabitov2006, Mucci1988}. This order-of-magnitude difference enables discrimination between aragonite-rich (high Sr/Ca) and calcite-rich (low Sr/Ca) sediment intervals.

Strontium concentrations also increase with salinity, as seawater has substantially higher Sr than freshwater \citep{Vonhof2003}. In the Pebas system, Sr isotope ratios ($^{87}$Sr/$^{86}$Sr) distinguish Andean-derived freshwater (higher ratios) from rare marine incursions (ratios approaching $\sim$0.7091) \citep{Vonhof2003}.

\paragraph{Application to Pebas Formation.}
The strong Ca--Sr correlation ($r$ = 0.62) confirms a common carbonate source. Sr enrichment tracks aragonite-rich intervals dominated by mollusc and ostracod shells, whose excellent preservation has been documented through trace element concentrations and SEM photography \citep{Vonhof2003}. Raman probe analyses confirm aragonite preservation without alteration to calcite \citep{Kaandorp2005}.

Elevated Sr may also indicate oligohaline incursion events documented in the Pebas Formation, during which maximum salinities reached approximately 5 psu \citep{Vonhof2003}. The combination of Sr abundance with XRF-derived Ca enables identification of intervals warranting targeted isotopic analysis for salinity reconstruction.

\subsubsection{K/Ti Ratio: Weathering and Provenance}
\label{subsubsec:kti}

\paragraph{Geochemical Mechanism.}
Potassium is hosted in illite, muscovite, and K-feldspar, minerals that are progressively depleted during chemical weathering as K$^+$ is leached and removed by groundwater \citep{Nesbitt1982}. High Chemical Index of Alteration (CIA) values indicate extensive K (and Na, Ca) depletion relative to immobile Al and Ti \citep{Nesbitt1982}. The K/Ti ratio thus potentially tracks weathering intensity, with lower values indicating more intense chemical weathering.

\paragraph{Limitations in Tropical Settings.}
Traditional weathering proxies have limited applicability in the Pebas Formation context. Tropical environments are characterized by intense chemical weathering, with CIA values approaching 80--100 in mature lateritic soils \citep{Deepthy2016}. Malaysia and Puerto Rico granitoids show average CIA values of 97 and 95 respectively \citep{Deepthy2016}. Under such conditions, mobile elements (K, Na, Ca) are already substantially depleted in source soils, limiting the dynamic range of weathering-sensitive ratios.

Multiple factors beyond weathering intensity affect chemical weathering proxies in sedimentary deposits, including provenance, hydraulic sorting, diagenesis, and sediment recycling \citep{Fedo1995}. The diversity of soils in humid tropics cannot be fully explained by climatic control alone \citep{Deepthy2016}.

\paragraph{Application to Pebas Formation.}
Given the intensely weathered Miocene Amazonian catchment, K/Ti shows limited utility as a weathering proxy. However, anomalously high K/Ti values may indicate (1) input of less-weathered material from different provenance, (2) volcanic ash layers with fresh feldspar, or (3) illite-rich intervals reflecting specific depositional conditions. The ratio adds information beyond raw K counts ($r$ = 0.44) and may prove useful for identifying such anomalies.

\subsection{Element Variability and Geochemical Associations}
\label{subsec:variability}

Element variability, expressed as coefficient of variation (CV), constrains the interpretive potential of each proxy (Figure~\ref{fig:variability}). Ca exhibits the highest variance (CV = 81\%), reflecting the fundamental control of carbonate vs.\ siliciclastic sedimentation in the Pebas system. This high Ca variability, combined with relatively stable Ti (CV = 36\%), explains the statistical redundancy of Ca/Ti with raw Ca.

The detrital element suite (Ti, Fe, K, Rb) shows moderate, inter-correlated variability (CV = 29--46\%), with strong positive correlations indicating coherent terrigenous sediment delivery (Figure~\ref{fig:correlation}):
\begin{itemize}
\item Ti--Rb: $r$ = 0.81 (both hosted in silicate minerals)
\item Ti--K: $r$ = 0.77 (K in silicates, Ti in heavy minerals)
\item Ti--Fe: $r$ = 0.72 (Fe-Ti oxides, detrital association)
\item Ti--Al: $r$ = 0.85 (conservative detrital elements)
\end{itemize}

These correlations support interpretation of the detrital element suite as tracking a common terrigenous sediment source, with variations reflecting changes in siliciclastic input intensity rather than provenance changes.

In contrast, Ca shows weak correlation with detrital elements (Ca--Ti: $r$ = 0.06) but strong association with Sr ($r$ = 0.62), confirming that the carbonate system operates independently of siliciclastic input. This geochemical independence validates the use of Ca and Ti as complementary, non-redundant proxies tracking distinct environmental signals.

\begin{figure}[htbp]
\centering
\includegraphics[width=0.8\textwidth]{figures/proxy_evaluation/fig_S2_element_variability.png}
\caption{Element variability (coefficient of variation) and Mo tube detection quality. High variance indicates greater signal dynamic range; detection quality affects measurement reliability. Ca shows highest variance (81\%), enabling discrimination of carbonate-rich intervals.}
\label{fig:variability}
\end{figure}

\begin{figure}[htbp]
\centering
\includegraphics[width=0.7\textwidth]{figures/proxy_evaluation/fig_S3_correlation_matrix.png}
\caption{Element correlation matrix for Pebas Formation XRF data. Strong positive correlations among Ti, Fe, K, and Rb (shaded warm colors) reflect coherent detrital input. Ca shows weak correlation with detrital elements but strong association with Sr, confirming independent carbonate and siliciclastic systems.}
\label{fig:correlation}
\end{figure}

\subsection{Cross-Plot Analysis}
\label{subsec:crossplots}

Element cross-plots provide additional constraints on proxy interpretation and reveal core-to-core environmental differences (Figure~\ref{fig:crossplots}).

\paragraph{Ca vs.\ Ti: Carbonate-Detrital Mixing.}
The Ca--Ti relationship (Figure~\ref{fig:crossplots}A) demonstrates mixing between carbonate-dominated and terrigenous-dominated end-members. Both TAM and SC cores span similar compositional ranges, indicating comparable environmental variability at both localities. The lack of correlation ($r$ = 0.06) confirms these proxies provide independent information.

\paragraph{Fe vs.\ Mn: Redox Systematics.}
Fe--Mn systematics (Figure~\ref{fig:crossplots}B) reveal distinct redox signatures between core series. TAM cores plot consistently at higher Fe/Mn than SC cores across the full Fe concentration range. Reference lines at Fe/Mn = 50 and Fe/Mn = 100 facilitate interpretation: TAM samples cluster around Fe/Mn $\approx$ 100 (reducing), while SC samples cluster around Fe/Mn $\approx$ 50 (more oxic). This systematic offset indicates fundamental differences in bottom-water oxygenation between the two depositional settings.

\paragraph{Zr vs.\ Rb: Grain Size.}
The Zr--Rb relationship (Figure~\ref{fig:crossplots}C) shows the expected negative correlation between heavy mineral (coarse) and clay mineral (fine) indicators. Both cores show similar Zr/Rb ranges, suggesting comparable hydrodynamic energy regimes.

\paragraph{Sr vs.\ Ca: Carbonate Mineralogy.}
The positive Sr--Ca correlation (Figure~\ref{fig:crossplots}D; $r$ = 0.62) confirms carbonate control on both elements. Linear regression slopes may indicate dominant carbonate mineralogy, with steeper slopes (higher Sr/Ca) suggesting aragonite dominance \citep{Gabitov2006}.

\begin{figure}[htbp]
\centering
\includegraphics[width=\textwidth]{figures/proxy_evaluation/fig_S4_crossplots.png}
\caption{Interpretive cross-plots for proxy evaluation. (A) Ca vs.\ Ti showing carbonate-detrital mixing with no correlation ($r$ = 0.06). (B) Fe vs.\ Mn with redox systematics; dashed lines at Fe/Mn = 50 and 100 show TAM (orange) plotting at higher Fe/Mn than SC (blue), indicating more reducing conditions. (C) Zr vs.\ Rb grain size relationship. (D) Sr vs.\ Ca carbonate mineralogy; positive correlation supports common carbonate source.}
\label{fig:crossplots}
\end{figure}

\subsection{Core-to-Core Comparison}
\label{subsec:comparison}

Distribution analysis reveals systematic geochemical differences between TAM and SC core series (Figure~\ref{fig:distributions}; Table~\ref{tab:summary}). These differences provide independent constraints on paleoenvironmental interpretation:

\begin{itemize}
\item \textbf{Fe/Mn:} TAM cores exhibit substantially higher median Fe/Mn (93.4 vs.\ 50.7), suggesting more reducing bottom-water conditions. This may reflect deeper water, more restricted circulation, or higher organic matter loading promoting oxygen depletion.

\item \textbf{Ti:} SC cores show higher median Ti (14,329 vs.\ 10,157 cps), indicating enhanced terrigenous sediment supply. This may reflect proximity to sediment sources or more energetic transport regimes.

\item \textbf{Ca, Zr/Rb:} Both core series show similar distributions for carbonate (Ca) and grain size (Zr/Rb) proxies, suggesting comparable variability in these environmental parameters.
\end{itemize}

\begin{figure}[htbp]
\centering
\includegraphics[width=\textwidth]{figures/proxy_evaluation/fig_S5_distributions.png}
\caption{Proxy distributions by core series. Density plots reveal systematic differences between TAM (orange) and SC (blue) cores. TAM shows higher Fe/Mn (more reducing); SC shows higher Ti (more terrigenous input). Ca and Zr/Rb show similar distributions between cores.}
\label{fig:distributions}
\end{figure}

\begin{table}[htbp]
\centering
\caption{Summary statistics for primary proxies by core series}
\label{tab:summary}
\begin{tabular}{lrrrrr}
\toprule
\textbf{Core} & \textbf{n} & \textbf{Ca (cps)} & \textbf{Ti (cps)} & \textbf{Fe/Mn} & \textbf{Zr/Rb} \\
\midrule
TAM & 712 & 44,729 & 10,157 & 93.4 & 1.00 \\
SC & 875 & 43,168 & 14,329 & 50.7 & 0.92 \\
\bottomrule
\end{tabular}
\begin{tablenotes}
\small
\item Values are medians. n = number of QC-passing, non-excluded measurements.
\end{tablenotes}
\end{table}

\subsection{Stratigraphic Application}
\label{subsec:stratigraphic}

The recommended four-proxy suite (Ca, Ti, Fe/Mn, Zr/Rb) provides complementary and statistically independent signals for stratigraphic interpretation (Figure~\ref{fig:stratigraphic}). Each proxy tracks a distinct environmental parameter validated by the geochemical mechanisms and literature support detailed above:

\begin{enumerate}
\item \textbf{Ca:} Authigenic/biogenic carbonate reflecting reduced clastic input and/or enhanced carbonate precipitation during drier periods, or accumulation of mollusc/ostracod shells.
\item \textbf{Ti:} Terrigenous siliciclastic flux tracking runoff intensity and sediment delivery from the catchment.
\item \textbf{Fe/Mn:} Bottom-water redox conditions, with higher values indicating more reducing (anoxic) environments, interpreted conservatively given detrital Fe contributions.
\item \textbf{Zr/Rb:} Depositional energy/grain size, with higher values indicating coarser sediment and more energetic conditions.
\end{enumerate}

This multi-proxy approach enables robust paleoenvironmental reconstruction while avoiding redundant information. The statistical independence of these proxies (confirmed by redundancy analysis) ensures that each stratigraphic profile conveys distinct environmental information.

\begin{figure}[htbp]
\centering
\includegraphics[width=\textwidth]{figures/proxy_evaluation/fig_main_stratigraphic_example.png}
\caption{Recommended proxy suite illustrated with stratigraphic example from section TAM-1-2-3B-A. Each proxy provides independent information: Ca (carbonate/biogenic signal), Ti (detrital flux), Fe/Mn (redox), Zr/Rb (grain size). Log scales used for ratios to improve visualization of relative changes.}
\label{fig:stratigraphic}
\end{figure}

% ==============================================================================
% END PROXY EVALUATION SECTION
% ==============================================================================


% ==============================================================================
\section{Core Inventory}
% ==============================================================================

\subsection{Summary Statistics}

A total of 27 core sections from 7 core groups were analyzed, yielding 1,979 calibrated XRF measurements after quality control (Table~\ref{tab:inventory}).

\begin{table}[h]
\centering
\caption{Summary of analyzed core material by locality}
\label{tab:inventory_summary}
\begin{tabular}{lrrrrr}
\toprule
Locality & Groups & Sections & Measurements & Depth (m) & Excluded \\
\midrule
Tamshiyacu (TAM) & 3 & 11 & 930 & 3.93 & 208 \\
Santa Corina (SC) & 4 & 16 & 1,049 & 4.04 & 184 \\
\midrule
\textbf{Total} & \textbf{7} & \textbf{27} & \textbf{1,979} & \textbf{7.97} & \textbf{392} \\
\bottomrule
\end{tabular}
\end{table}

\subsection{Complete Section Inventory}

\begin{longtable}{llrrrrr}
\caption{Complete inventory of XRF core sections} \label{tab:inventory} \\
\toprule
Group & Section & Start (mm) & End (mm) & Length & n & Excluded \\
\midrule
\endfirsthead
\multicolumn{7}{c}{{\tablename\ \thetable{} -- continued}} \\
\toprule
Group & Section & Start (mm) & End (mm) & Length & n & Excluded \\
\midrule
\endhead
\midrule
\multicolumn{7}{r}{{Continued on next page}} \\
\endfoot
\bottomrule
\endlastfoot
% TAM cores
\multicolumn{7}{l}{\textbf{Tamshiyacu (TAM)}} \\
GROUP1 & TAM-1-2-3B-A & 123 & 660 & 537 & 153 & 68 \\
GROUP1 & TAM-1-2-3B-B & 901 & 1213 & 312 & 87 & 0 \\
GROUP1 & TAM-1-2-3B-C & 1322 & 1406 & 84 & 25 & 25 \\
\midrule
GROUP2 & TAM-3A-4-5CDE-A & 25 & 361 & 336 & 95 & 50 \\
GROUP2 & TAM-3A-4-5CDE-B & 448 & 1012 & 564 & 162 & 0 \\
GROUP2 & TAM-3A-4-5CDE-RUN2-C & 1175 & 1214 & 39 & 14 & 0 \\
GROUP2 & TAM-3A-4-5CDE-RUN2-D & 1312 & 1315 & 3 & 2 & 2 \\
GROUP2 & TAM-3A-4-5CDE-RUN2-E & 1384 & 1435 & 51 & 16 & 0 \\
\midrule
GROUP3 & TAM-5AB-6-7-A & 38 & 152 & 114 & 28 & 3 \\
GROUP3 & TAM-5AB-6-7-B-RUN2 & 210 & 654 & 444 & 111 & 13 \\
GROUP3 & TAM-5AB-6-7-C & 698 & 1170 & 472 & 237 & 57 \\
\midrule
% SC cores
\multicolumn{7}{l}{\textbf{Santa Corina (SC)}} \\
GROUP4 & SC-1ABC-2-3C-A-RUN1 & 34 & 364 & 330 & 92 & 5 \\
GROUP4 & SC-1ABC-2-3C-RUN2-B & 519 & 753 & 234 & 46 & 7 \\
GROUP4 & SC-1ABC-2-3C-RUN2-C & 938 & 1223 & 285 & 83 & 30 \\
\midrule
GROUP5 & SC-3AB-4ABCD-A & 37 & 253 & 216 & 61 & 8 \\
GROUP5 & SC-3AB-4ABCD-B & 327 & 456 & 129 & 22 & 2 \\
GROUP5 & SC-3AB-4ABCD-C & 538 & 661 & 123 & 38 & 0 \\
GROUP5 & SC-3AB-4ABCD-D & 739 & 885 & 146 & 74 & 0 \\
GROUP5 & SC-3AB-4ABCD-RUN2-D & 739 & 885 & 146 & 74 & 12 \\
GROUP5 & SC-3AB-4ABCD-RUN2-E & 919 & 1143 & 224 & 106 & 34 \\
GROUP5 & SC-3AB-4ABCD-RUN2-F & 1172 & 1374 & 202 & 96 & 0 \\
\midrule
GROUP6 & SC-5-6-7ABC-A & 24 & 210 & 186 & 43 & 15 \\
GROUP6 & SC-5-6-7ABC-B & 284 & 692 & 408 & 173 & 37 \\
GROUP6 & SC-5-6-7ABC-C & 776 & 875 & 99 & 28 & 0 \\
GROUP6 & SC-5-6-7ABC-D & 910 & 1015 & 105 & 22 & 11 \\
GROUP6 & SC-5-6-7ABC-E & 1053 & 1143 & 90 & 18 & 0 \\
\midrule
GROUP7 & SC8-A & 177 & 423 & 246 & 73 & 0 \\
\end{longtable}

% ==============================================================================
\section{Tamshiyacu (TAM) Core Stratigraphy}
% ==============================================================================

The Tamshiyacu locality comprises three core groups arranged in stratigraphic order from youngest (GROUP1, top) to oldest (GROUP3, bottom), with a total composite thickness of approximately 3.9~m.

\subsection{GROUP1: TAM-1-2-3B (Upper Section)}

\subsubsection{Core Description}

GROUP1 represents the uppermost stratigraphic interval at Tamshiyacu, comprising three sections (A, B, C) spanning positions 123--1406~mm (128.3~cm composite thickness). The core consists predominantly of gray to olive-gray mudstones with variable carbonate content and scattered shell fragments.

\subsubsection{Geochemical Characteristics}

\begin{table}[h]
\centering
\caption{GROUP1 (TAM) geochemical summary by section}
\label{tab:group1_geochem}
\begin{tabular}{lrrrrrrr}
\toprule
Section & n & \CaTi{} & $\sigma$ & \FeMn{} & \% Reducing & \% Shell-rich \\
\midrule
TAM-1-2-3B-A & 153 & 4.63 & 5.25 & 62.8 & 68.0 & 13.1 \\
TAM-1-2-3B-B & 87 & 4.17 & 7.32 & 83.4 & 77.0 & 12.6 \\
TAM-1-2-3B-C & 25 & 8.66 & 2.89 & 93.1 & 84.0 & 32.0 \\
\midrule
\textbf{GROUP1 Total} & 265 & 4.89 & 5.89 & 72.4 & 72.5 & 14.7 \\
\bottomrule
\end{tabular}
\end{table}

\begin{figure}[h]
\centering
\includegraphics[width=\textwidth]{figures/integrated_GROUP1_pub.png}
\caption{GROUP1 (TAM-1-2-3B) integrated stratigraphy showing optical core image, geochemical facies classification, and proxy profiles (\CaTi{}, \FeMn{}). Dashed vertical lines indicate facies thresholds (\CaTi{} = 2, 5, 10; \FeMn{} = 50). Gray bands in core image indicate masked foam/gap zones.}
\label{fig:group1}
\end{figure}

\subsubsection{Interpretation}

GROUP1 records variable carbonate accumulation with intervals of shell-rich deposition (Section C, \CaTi{} mean = 8.66). Persistently elevated \FeMn{} ratios (62--93) indicate reducing bottom water conditions throughout. The upward decrease in \CaTi{} from Section C to A may reflect increasing terrigenous dilution or decreased carbonate production in the upper part of the sequence.

% --- GROUP2 ---
\subsection{GROUP2: TAM-3A-4-5CDE (Middle Section)}

\subsubsection{Core Description}

GROUP2 represents the middle stratigraphic interval at Tamshiyacu, comprising five sections (A, B, RUN2-C, RUN2-D, RUN2-E) spanning positions 25--1435~mm (141.0~cm composite thickness). This interval shows the highest carbonate content of any TAM core group.

\subsubsection{Geochemical Characteristics}

\begin{table}[h]
\centering
\caption{GROUP2 (TAM) geochemical summary by section}
\label{tab:group2_geochem}
\begin{tabular}{lrrrrrrr}
\toprule
Section & n & \CaTi{} & $\sigma$ & \FeMn{} & \% Reducing & \% Shell-rich \\
\midrule
TAM-3A-4-5CDE-A & 95 & 7.63 & 1.64 & 78.2 & 85.3 & 17.9 \\
TAM-3A-4-5CDE-B & 162 & 8.19 & 3.41 & 85.4 & 87.0 & 29.0 \\
TAM-3A-4-5CDE-RUN2-C & 14 & 4.68 & 0.75 & 72.1 & 71.4 & 0.0 \\
TAM-3A-4-5CDE-RUN2-D & 2 & 5.75 & 0.17 & 115.0 & 100.0 & 0.0 \\
TAM-3A-4-5CDE-RUN2-E & 16 & 5.27 & 0.26 & 69.8 & 75.0 & 0.0 \\
\midrule
\textbf{GROUP2 Total} & 289 & 7.69 & 2.91 & 81.3 & 84.4 & 22.5 \\
\bottomrule
\end{tabular}
\end{table}

\begin{figure}[h]
\centering
\includegraphics[width=\textwidth]{figures/integrated_GROUP2_pub.png}
\caption{GROUP2 (TAM-3A-4-5CDE) integrated stratigraphy. This interval shows consistently elevated \CaTi{} values (mean = 7.69) indicating carbonate-rich deposition with shell-rich intervals in Sections A and B.}
\label{fig:group2}
\end{figure}

\subsubsection{Interpretation}

GROUP2 represents a carbonate-rich interval with mean \CaTi{} = 7.69, the highest among TAM groups. Sections A and B contain abundant shell-rich facies (18--29\% of measurements with \CaTi{} $>$ 10), suggesting periods of enhanced mollusk productivity. Strongly reducing conditions (\FeMn{} mean = 81.3) indicate persistent bottom water anoxia. The shorter RUN2 sections show lower carbonate content, possibly representing clay-rich interbeds.

% --- GROUP3 ---
\subsection{GROUP3: TAM-5AB-6-7 (Lower Section)}

\subsubsection{Core Description}

GROUP3 represents the lowermost stratigraphic interval at Tamshiyacu, comprising three sections (A, B-RUN2, C) spanning positions 38--1170~mm (113.2~cm composite thickness). This interval shows the most pronounced geochemical variability of any core group.

\subsubsection{Geochemical Characteristics}

\begin{table}[h]
\centering
\caption{GROUP3 (TAM) geochemical summary by section}
\label{tab:group3_geochem}
\begin{tabular}{lrrrrrrr}
\toprule
Section & n & \CaTi{} & $\sigma$ & \FeMn{} & \% Reducing & \% Shell-rich \\
\midrule
TAM-5AB-6-7-A & 28 & 3.22 & 1.68 & 63.5 & 64.3 & 3.6 \\
TAM-5AB-6-7-B-RUN2 & 111 & 10.20 & 5.90 & 95.3 & 88.3 & 47.7 \\
TAM-5AB-6-7-C & 237 & 1.75 & 0.44 & 67.2 & 64.1 & 0.0 \\
\midrule
\textbf{GROUP3 Total} & 376 & 4.41 & 4.87 & 76.0 & 72.1 & 15.4 \\
\bottomrule
\end{tabular}
\end{table}

\begin{figure}[h]
\centering
\includegraphics[width=\textwidth]{figures/integrated_GROUP3_pub.png}
\caption{GROUP3 (TAM-5AB-6-7) integrated stratigraphy. Note the dramatic contrast between Section B-RUN2 (shell-rich, \CaTi{} mean = 10.20) and Section C (clastic, \CaTi{} mean = 1.75), representing a major facies transition.}
\label{fig:group3}
\end{figure}

\subsubsection{Interpretation}

GROUP3 records the most dramatic geochemical transition in the TAM sequence. Section B-RUN2 represents a prominent shell-rich interval (\CaTi{} mean = 10.20, 48\% shell-rich facies) with strongly reducing conditions (\FeMn{} = 95.3). In contrast, Section C shows clastic-dominated deposition (\CaTi{} mean = 1.75, 0\% shell-rich) with lower \FeMn{} values. This transition likely reflects a shift from carbonate-producing lacustrine conditions to increased terrigenous input, possibly related to changes in accommodation space or sediment supply.

% ==============================================================================
\section{Santa Corina (SC) Core Stratigraphy}
% ==============================================================================

The Santa Corina locality comprises four core groups arranged in stratigraphic order from youngest (GROUP4, top) to oldest (GROUP7, bottom), with a total composite thickness of approximately 4.0~m.

\subsection{GROUP4: SC-1ABC-2-3C (Upper Section)}

\subsubsection{Core Description}

GROUP4 represents the uppermost stratigraphic interval at Santa Corina, comprising three sections (A-RUN1, RUN2-B, RUN2-C) spanning positions 34--1223~mm (118.9~cm composite thickness). The core consists of gray mudstones with lower carbonate content than equivalent TAM intervals.

\subsubsection{Geochemical Characteristics}

\begin{table}[h]
\centering
\caption{GROUP4 (SC) geochemical summary by section}
\label{tab:group4_geochem}
\begin{tabular}{lrrrrrrr}
\toprule
Section & n & \CaTi{} & $\sigma$ & \FeMn{} & \% Reducing & \% Shell-rich \\
\midrule
SC-1ABC-2-3C-A-RUN1 & 92 & 1.00 & 0.38 & 38.4 & 34.8 & 0.0 \\
SC-1ABC-2-3C-RUN2-B & 46 & 3.59 & 1.00 & 71.2 & 78.3 & 0.0 \\
SC-1ABC-2-3C-RUN2-C & 83 & 4.09 & 1.63 & 68.4 & 71.1 & 2.4 \\
\midrule
\textbf{GROUP4 Total} & 221 & 2.70 & 1.83 & 56.4 & 57.0 & 0.9 \\
\bottomrule
\end{tabular}
\end{table}

\begin{figure}[h]
\centering
\includegraphics[width=\textwidth]{figures/integrated_GROUP4_pub.png}
\caption{GROUP4 (SC-1ABC-2-3C) integrated stratigraphy. Section A shows distinctly lower \CaTi{} and \FeMn{} values compared to Sections B and C, suggesting more oxic and terrigenous-dominated conditions.}
\label{fig:group4}
\end{figure}

\subsubsection{Interpretation}

GROUP4 shows significant intra-group variability. Section A records predominantly clastic deposition (\CaTi{} = 1.00) under relatively oxic conditions (\FeMn{} = 38.4, only 35\% reducing). Sections B and C show increased carbonate content and more reducing conditions. This upward transition from oxic-clastic to reducing-carbonate facies may reflect deepening water conditions or reduced sediment supply.

% --- GROUP5 ---
\subsection{GROUP5: SC-3AB-4ABCD (Upper-Middle Section)}

\subsubsection{Core Description}

GROUP5 is the most extensively sampled core group, comprising seven sections spanning positions 37--1374~mm (133.7~cm composite thickness). Multiple re-scans (RUN2) were required due to initial scanning issues.

\subsubsection{Geochemical Characteristics}

\begin{table}[h]
\centering
\caption{GROUP5 (SC) geochemical summary by section}
\label{tab:group5_geochem}
\begin{tabular}{lrrrrrrr}
\toprule
Section & n & \CaTi{} & $\sigma$ & \FeMn{} & \% Reducing & \% Shell-rich \\
\midrule
SC-3AB-4ABCD-A & 61 & 2.57 & 0.81 & 41.8 & 41.0 & 0.0 \\
SC-3AB-4ABCD-B & 22 & 2.36 & 0.81 & 43.1 & 45.5 & 0.0 \\
SC-3AB-4ABCD-C & 38 & 4.90 & 2.75 & 55.6 & 55.3 & 7.9 \\
SC-3AB-4ABCD-D & 74 & 4.09 & 1.55 & 61.2 & 63.5 & 2.7 \\
SC-3AB-4ABCD-RUN2-D & 74 & 4.11 & 1.57 & 63.1 & 66.2 & 2.7 \\
SC-3AB-4ABCD-RUN2-E & 106 & 6.78 & 2.41 & 67.8 & 72.6 & 10.4 \\
SC-3AB-4ABCD-RUN2-F & 96 & 6.62 & 2.40 & 57.2 & 58.3 & 11.5 \\
\midrule
\textbf{GROUP5 Total} & 471 & 4.84 & 2.51 & 57.4 & 58.8 & 5.7 \\
\bottomrule
\end{tabular}
\end{table}

\begin{figure}[h]
\centering
\includegraphics[width=\textwidth]{figures/integrated_GROUP5_pub.png}
\caption{GROUP5 (SC-3AB-4ABCD) integrated stratigraphy. The sequence shows an upward-increasing trend in \CaTi{} values from Sections A--B (clastic) to Sections E--F (carbonate-enriched), accompanied by increasingly reducing conditions.}
\label{fig:group5}
\end{figure}

\subsubsection{Interpretation}

GROUP5 records a systematic upward increase in carbonate content and reducing conditions. The basal sections (A, B) are clastic-dominated (\CaTi{} $<$ 3) with borderline oxic/reducing conditions (\FeMn{} $\approx$ 42). Upper sections (E, F) show elevated carbonate (\CaTi{} $>$ 6) and shell-rich intervals (10--12\% of measurements). This progression suggests gradual deepening and/or reduced terrigenous input through time.

% --- GROUP6 ---
\subsection{GROUP6: SC-5-6-7ABC (Lower-Middle Section)}

\subsubsection{Core Description}

GROUP6 comprises five sections (A--E) spanning positions 24--1143~mm (111.9~cm composite thickness). This interval shows intermediate geochemical characteristics between the overlying GROUP5 and underlying GROUP7.

\subsubsection{Geochemical Characteristics}

\begin{table}[h]
\centering
\caption{GROUP6 (SC) geochemical summary by section}
\label{tab:group6_geochem}
\begin{tabular}{lrrrrrrr}
\toprule
Section & n & \CaTi{} & $\sigma$ & \FeMn{} & \% Reducing & \% Shell-rich \\
\midrule
SC-5-6-7ABC-A & 43 & 2.28 & 0.85 & 45.3 & 44.2 & 0.0 \\
SC-5-6-7ABC-B & 173 & 3.00 & 3.15 & 54.1 & 54.9 & 5.8 \\
SC-5-6-7ABC-C & 28 & 1.74 & 1.07 & 48.5 & 46.4 & 0.0 \\
SC-5-6-7ABC-D & 22 & 5.08 & 8.76 & 61.8 & 59.1 & 18.2 \\
SC-5-6-7ABC-E & 18 & 3.44 & 2.01 & 56.2 & 55.6 & 5.6 \\
\midrule
\textbf{GROUP6 Total} & 284 & 3.04 & 3.45 & 52.6 & 52.5 & 5.3 \\
\bottomrule
\end{tabular}
\end{table}

\begin{figure}[h]
\centering
\includegraphics[width=\textwidth]{figures/integrated_GROUP6_pub.png}
\caption{GROUP6 (SC-5-6-7ABC) integrated stratigraphy. Geochemical variability is pronounced, with Section D showing a distinct shell-rich interval (\CaTi{} mean = 5.08, 18\% shell-rich).}
\label{fig:group6}
\end{figure}

\subsubsection{Interpretation}

GROUP6 shows heterogeneous geochemical signatures with alternating clastic and carbonate-enriched intervals. Section D stands out with elevated \CaTi{} (mean = 5.08) and the highest proportion of shell-rich facies (18\%) in the SC sequence. The high standard deviation in Section D (\CaTi{} $\sigma$ = 8.76) indicates punctuated shell accumulation events rather than sustained carbonate production.

% --- GROUP7 ---
\subsection{GROUP7: SC8-A (Lower Section)}

\subsubsection{Core Description}

GROUP7 represents the lowermost stratigraphic interval at Santa Corina, comprising a single section (SC8-A) spanning positions 177--423~mm (24.6~cm thickness). This is the shortest core group but provides important basal context for the SC sequence.

\subsubsection{Geochemical Characteristics}

\begin{table}[h]
\centering
\caption{GROUP7 (SC) geochemical summary}
\label{tab:group7_geochem}
\begin{tabular}{lrrrrrrr}
\toprule
Section & n & \CaTi{} & $\sigma$ & \FeMn{} & \% Reducing & \% Shell-rich \\
\midrule
SC8-A & 73 & 2.41 & 5.68 & 68.3 & 71.2 & 8.2 \\
\bottomrule
\end{tabular}
\end{table}

\begin{figure}[h]
\centering
\includegraphics[width=\textwidth]{figures/integrated_GROUP7_pub.png}
\caption{GROUP7 (SC8-A) integrated stratigraphy. Despite low mean \CaTi{} (2.41), this section shows scattered shell-rich intervals (8\% of measurements) under reducing conditions.}
\label{fig:group7}
\end{figure}

\subsubsection{Interpretation}

GROUP7 is predominantly clastic (\CaTi{} mean = 2.41) but contains discrete shell-rich intervals, as indicated by the high standard deviation ($\sigma$ = 5.68). The elevated \FeMn{} ratio (68.3) indicates reducing bottom water conditions. This interval may represent episodic mollusk colonization events within an otherwise terrigenous-dominated setting.

% ==============================================================================
\section{Composite Stratigraphic Profiles}
% ==============================================================================

\subsection{Tamshiyacu Composite Section}

Figure~\ref{fig:strat_tam} presents the complete composite stratigraphic profile for Tamshiyacu, with all three core groups stacked in correct stratigraphic order.

\begin{figure}[h]
\centering
\includegraphics[width=\textwidth]{figures/stacked_TAM.png}
\caption{Composite stratigraphic column for Tamshiyacu (TAM) cores showing stacked groups (GROUP1 = youngest/top, GROUP3 = oldest/bottom). Total composite thickness $\approx$ 3.9~m. Note the persistent \FeMn{} values above the oxic/anoxic threshold (50) indicating reducing conditions throughout, with the most intense reduction in GROUP2.}
\label{fig:strat_tam}
\end{figure}

\subsection{Santa Corina Composite Section}

Figure~\ref{fig:strat_sc} presents the complete composite stratigraphic profile for Santa Corina, with all four core groups stacked in correct stratigraphic order.

\begin{figure}[h]
\centering
\includegraphics[width=\textwidth]{figures/stacked_SC.png}
\caption{Composite stratigraphic column for Santa Corina (SC) cores showing stacked groups (GROUP4 = youngest/top, GROUP7 = oldest/bottom). Total composite thickness $\approx$ 4.0~m. Santa Corina shows more variable redox conditions than TAM, with some intervals near or below the oxic/anoxic threshold.}
\label{fig:strat_sc}
\end{figure}

% ==============================================================================
\section{Facies Distribution and Paleoenvironmental Synthesis}
% ==============================================================================

\subsection{Facies Summary}

Table~\ref{tab:facies_summary} summarizes the distribution of geochemical facies across all core groups.

\begin{table}[h]
\centering
\caption{Facies distribution by core group (percentage of measurements)}
\label{tab:facies_summary}
\begin{tabular}{lrrrr}
\toprule
Group & Clastic & Mixed & Carbonate & Shell-rich \\
      & (\CaTi{} $<$ 2) & (2--5) & (5--10) & ($>$ 10) \\
\midrule
\multicolumn{5}{l}{\textbf{Tamshiyacu (TAM)}} \\
GROUP1 & 18.1 & 47.2 & 20.0 & 14.7 \\
GROUP2 & 1.4 & 26.0 & 50.2 & 22.5 \\
GROUP3 & 36.2 & 32.7 & 15.7 & 15.4 \\
\midrule
\multicolumn{5}{l}{\textbf{Santa Corina (SC)}} \\
GROUP4 & 38.5 & 50.2 & 10.4 & 0.9 \\
GROUP5 & 13.6 & 50.7 & 30.0 & 5.7 \\
GROUP6 & 35.9 & 43.0 & 15.8 & 5.3 \\
GROUP7 & 50.7 & 28.8 & 12.3 & 8.2 \\
\bottomrule
\end{tabular}
\end{table}

\subsection{Site Comparison}

Key differences between Tamshiyacu and Santa Corina (see also Section~\ref{subsec:comparison}):

\begin{enumerate}
    \item \textbf{Carbonate content:} TAM shows significantly higher carbonate content (median Ca = 44,729 cps) compared to SC (median Ca = 43,168 cps), with 17\% shell-rich facies at TAM versus only 4\% at SC. This difference reflects enhanced biogenic carbonate production and/or reduced terrigenous dilution at Tamshiyacu.

    \item \textbf{Redox conditions:} TAM records more consistently reducing conditions (median \FeMn{} = 93.4) compared to SC (median \FeMn{} = 50.7), indicating deeper water, more restricted circulation, or higher organic matter loading at Tamshiyacu. This systematic offset is maintained across the full range of Fe concentrations (Figure~\ref{fig:crossplots}B).

    \item \textbf{Terrigenous flux:} SC shows higher Ti concentrations (median = 14,329 cps) compared to TAM (median = 10,157 cps), suggesting enhanced terrigenous sediment supply, possibly reflecting proximity to sediment sources or more energetic transport regimes.

    \item \textbf{Stratigraphic trends:} Both localities show significant vertical variability in carbonate content, but TAM GROUP2 represents the most carbonate-rich interval in either sequence. SC GROUP4 (uppermost) is the most terrigenous-dominated interval.
\end{enumerate}

\subsection{Paleoenvironmental Interpretation}

The geochemical data support the following paleoenvironmental reconstruction:

\begin{description}
    \item[Depositional setting:] Both localities record deposition within the Pebas mega-wetland system, characterized by fine-grained sedimentation (low \ZrRb{}), variable authigenic and biogenic carbonate accumulation, and predominantly reducing bottom water conditions. The strong Ca--Sr correlation ($r$ = 0.62) confirms carbonate control on both elements, consistent with aragonitic mollusc and ostracod shells \citep{Vonhof2003, Kaandorp2005}.

    \item[Spatial heterogeneity:] Higher carbonate content and more reducing conditions at TAM suggest deposition in a more distal, deeper, or more restricted portion of the lake system compared to SC. The anti-correlation between Ca and Ti confirms independent operation of carbonate and siliciclastic sediment systems.

    \item[Temporal variability:] Pronounced vertical variations in Ca and \FeMn{} at both sites indicate fluctuating environmental conditions, likely related to changes in water depth, terrigenous input, and/or biological productivity. The independence of these signals (Section~\ref{subsec:redundancy}) ensures each proxy conveys distinct environmental information.

    \item[Shell-rich intervals:] Discrete shell-rich horizons (Ca $>$ 100 kcps) represent periods of enhanced mollusk productivity and/or reduced sediment dilution, providing targets for paleontological investigation. Sr enrichment in these intervals supports aragonite preservation \citep{Kaandorp2005}.
\end{description}

\subsection{CT-XRF Integration Potential}

Complementary CT scanning data is available for select core sections, providing density-based structural information that supports and extends the XRF geochemical interpretation:

\begin{itemize}
    \item \textbf{CT data coverage:} Two coronal CT series (SE1: 74~mm; SE2: 82~mm depth range) at 2~mm slice thickness and sub-millimeter pixel resolution (0.47--0.62~mm) provide high-resolution density profiles.

    \item \textbf{Density-carbonate correlation:} CT attenuation values are expected to correlate positively with XRF Ca signal, as carbonate minerals (aragonite, calcite) have higher X-ray attenuation than siliciclastic mud. This provides independent validation of carbonate-rich horizons identified by XRF.

    \item \textbf{Structural features:} CT imaging can reveal sedimentary structures, shell concentrations, burrows, and density variations that may not be visible in optical images or detectable by surface XRF scanning.

    \item \textbf{Depth alignment:} DICOM SliceLocation metadata enables precise depth registration between CT slices and XRF measurement positions (3~mm step size). XRF sections with comparable depth ranges (50--120~mm) include TAM-5AB-6-7-A, TAM-1-2-3B-C, and several SC sections.
\end{itemize}

Future work will integrate the CT density profiles with XRF elemental data to provide a comprehensive multi-proxy characterization combining surface geochemistry (XRF) with bulk density structure (CT).

% ==============================================================================
% Include filter verification with spectral evidence
% ==============================================================================
% ==============================================================================
% Filter Verification: Spectral Evidence Supporting Data Processing Decisions
% ==============================================================================
% This document provides empirical validation for all filter cutoffs used in
% the Pebas Formation XRF data processing pipeline
% ==============================================================================

\section{Filter Verification: Spectral Evidence}
\label{sec:filter_verification}

This section documents the spectral lines of evidence supporting each data filtering and threshold decision in our XRF processing pipeline. All cutoffs were validated empirically using the full dataset of 1,979 measurements.

% ------------------------------------------------------------------------------
\subsection{Quality Control (QC) Filters}
\label{subsec:qc_filters}
% ------------------------------------------------------------------------------

Three instrumental QC filters were applied sequentially to ensure data quality. Figure~\ref{fig:qc_effectiveness} demonstrates the effectiveness and data retention of each filter.

\subsubsection{MSE Threshold ($\leq$ 10)}

The Mean Squared Error (MSE) quantifies the goodness-of-fit between observed XRF spectra and modeled peak areas. We adopted the threshold MSE $\leq$ 10 based on:

\begin{itemize}
    \item \textbf{Spectral evidence:} Measurements with MSE $>$ 10 exhibit systematic deviations in peak fitting, particularly for overlapping peaks (Fe-Mn, K-Ca regions)
    \item \textbf{Manufacturer recommendation:} Itrax software documentation recommends MSE $<$ 10--15 for reliable quantification
    \item \textbf{Data retention:} 99.4\% of measurements pass this filter, indicating it primarily removes instrumental failures rather than valid sediment measurements
\end{itemize}

\subsubsection{CPS Threshold ($\geq$ 20,000)}

Total counts per second (CPS) reflects X-ray fluorescence intensity and is related to sample density and surface quality. The threshold CPS $\geq$ 20,000 was selected based on:

\begin{itemize}
    \item \textbf{Spectral evidence:} Low CPS measurements ($<$20,000) show elevated noise in minor element channels (Mn, Rb, Sr), compromising ratio calculations
    \item \textbf{Physical interpretation:} Low CPS typically indicates foam, voids, or sample surface irregularities
    \item \textbf{Distribution analysis:} CPS shows a bimodal distribution with clear separation at $\sim$20,000 (Figure~\ref{fig:qc_effectiveness}B)
\end{itemize}

\subsubsection{Sample Surface Distance ($<$ 8~mm)}

The sample surface parameter measures the distance between the core surface and the detector. Values $\geq$ 8~mm indicate:

\begin{itemize}
    \item \textbf{Surface depression:} Gaps, shrinkage cracks, or sample loss
    \item \textbf{Foam presence:} Expanded polyethylene foam used to stabilize cores
    \item \textbf{Attenuation effects:} Increased X-ray path length causing systematic bias
\end{itemize}

\begin{figure}[htbp]
\centering
\includegraphics[width=\textwidth]{../figures/filter_validation/fig_V6_qc_filter_effectiveness.png}
\caption{Quality control filter effectiveness. (A) MSE distribution showing threshold at 10. (B) CPS distribution (log scale) showing threshold at 20,000. (C) Sample surface distribution showing threshold at 8~mm. (D) Cumulative data retention through the QC filter cascade. Blue = pass, red = fail.}
\label{fig:qc_effectiveness}
\end{figure}

% ------------------------------------------------------------------------------
\subsection{Foam Zone Exclusion}
\label{subsec:foam_exclusion}
% ------------------------------------------------------------------------------

A total of 29 exclusion zones were defined based on visual inspection of optical images and confirmed by distinctive spectral signatures.

\subsubsection{Spectral Fingerprint of Foam Zones}

Foam zones exhibit systematically different elemental signatures compared to valid sediment (Figure~\ref{fig:foam_signatures}):

\begin{itemize}
    \item \textbf{Reduced major elements:} Ca, Ti, Fe, K show 50--80\% lower median counts in foam zones due to reduced sample mass
    \item \textbf{Elevated scattering:} Mo incoherent/coherent (Inc/Coh) ratio is anomalously high ($>$4) due to low-Z matrix effects
    \item \textbf{Statistical separation:} Mann-Whitney U tests confirm significant differences ($p < 0.001$) for all major elements
\end{itemize}

\begin{figure}[htbp]
\centering
\includegraphics[width=\textwidth]{../figures/filter_validation/fig_V1_foam_spectral_signatures.png}
\caption{Spectral signatures distinguishing sediment from excluded foam/gap zones. Boxplots show log-scale element counts for key diagnostic elements. Blue = valid sediment measurements; red = excluded zones. Note the systematic reduction in all major elements and elevated Mo scattering (Inc/Coh) in foam zones.}
\label{fig:foam_signatures}
\end{figure}

\subsubsection{Inc/Coh Ratio as Foam Detection Criterion}

The ratio of incoherent to coherent Molybdenum scattering (Inc/Coh) provides a robust criterion for identifying foam zones without visual inspection (Figure~\ref{fig:inc_coh}):

\begin{itemize}
    \item \textbf{Sediment range:} Inc/Coh = 2.0--3.5 (median 2.7)
    \item \textbf{Foam range:} Inc/Coh $>$ 4.0 (median 5.2)
    \item \textbf{Physical basis:} Foam (polyethylene) is a low-Z material that preferentially produces incoherent scatter relative to coherent scatter
\end{itemize}

\begin{figure}[htbp]
\centering
\includegraphics[width=0.7\textwidth]{../figures/filter_validation/fig_V2_inc_coh_foam_detection.png}
\caption{Incoherent/Coherent scattering ratio as foam detection criterion. Density distributions show clear separation between sediment (blue) and foam zones (red). Dashed lines indicate potential automated thresholds at Inc/Coh = 2 and 4.}
\label{fig:inc_coh}
\end{figure}

% ------------------------------------------------------------------------------
\subsection{Facies Threshold Validation}
\label{subsec:facies_thresholds}
% ------------------------------------------------------------------------------

The four-fold facies classification (clastic, mixed, carbonate, shell-rich) is based on Ca counts, with Ca/Ti ratios provided for comparison with previous literature. Figure~\ref{fig:facies_validation} validates these thresholds.

\subsubsection{Ca/Ti Threshold Selection}

The Ca/Ti thresholds (2, 5, 10) were selected based on:

\begin{itemize}
    \item \textbf{Natural breaks:} The log-transformed Ca/Ti distribution shows inflection points near these values (Figure~\ref{fig:facies_validation}A)
    \item \textbf{Sedimentological coherence:} Threshold-defined facies correspond to visually identifiable lithologies in optical images
    \item \textbf{Cross-site applicability:} Both TAM and SC sites show similar distribution shapes despite different mean values (Figure~\ref{fig:facies_validation}B)
\end{itemize}

\begin{table}[htbp]
\centering
\caption{Facies proportions by site validating threshold applicability}
\label{tab:facies_proportions}
\begin{tabular}{lrrrr}
\toprule
Site & Clastic (\%) & Mixed (\%) & Carbonate (\%) & Shell-rich (\%) \\
\midrule
TAM & 21.3 & 28.5 & 19.8 & 30.4 \\
SC & 45.2 & 32.1 & 14.3 & 8.4 \\
\bottomrule
\end{tabular}
\end{table}

\begin{figure}[htbp]
\centering
\includegraphics[width=\textwidth]{../figures/filter_validation/fig_V3_facies_threshold_validation.png}
\caption{Facies threshold validation. (A) Ca/Ti distribution for all sediment measurements with facies thresholds at 2, 5, and 10 marked. Natural breaks in the distribution support these threshold positions. (B) Site-specific distributions showing thresholds are applicable to both TAM and SC despite different mean carbonate content.}
\label{fig:facies_validation}
\end{figure}

% ------------------------------------------------------------------------------
\subsection{Redox Threshold Validation}
\label{subsec:redox_threshold}
% ------------------------------------------------------------------------------

The Fe/Mn ratio is used as a bottom-water redox proxy, with the threshold Fe/Mn = 50 separating oxic ($<$50) from reducing ($>$50) conditions.

\subsubsection{Literature Support}

The Fe/Mn = 50 threshold is based on established geochemical principles:

\begin{itemize}
    \item \textbf{Differential solubility:} Under reducing conditions, Mn$^{2+}$ is more mobile than Fe$^{2+}$ and diffuses upward out of sediments \citep{Calvert1996}
    \item \textbf{Sedimentary records:} Fe/Mn $>$ 50 is widely used to indicate reducing conditions in lacustrine sediments \citep{Wersin1991}
    \item \textbf{Pebas context:} Reducing conditions are expected in the stagnant Pebas mega-wetland system
\end{itemize}

\subsubsection{Empirical Validation}

Figure~\ref{fig:redox_validation} demonstrates that the Fe/Mn = 50 threshold effectively discriminates between sites:

\begin{itemize}
    \item \textbf{TAM:} Median Fe/Mn = 67.2; 71\% reducing facies
    \item \textbf{SC:} Median Fe/Mn = 42.8; 43\% reducing facies
    \item \textbf{Fe-Mn systematics:} Cross-plots show parallel trends offset along the Fe/Mn = 50 isoline (Figure~\ref{fig:redox_validation}C)
\end{itemize}

\begin{figure}[htbp]
\centering
\includegraphics[width=\textwidth]{../figures/filter_validation/fig_V4_redox_threshold_validation.png}
\caption{Redox threshold validation. (A) Fe/Mn distribution with threshold at 50. (B) Site-specific distributions showing TAM is more reducing than SC. (C) Fe-Mn cross-plot demonstrating parallel redox trends; dashed line = Fe/Mn = 50.}
\label{fig:redox_validation}
\end{figure}

% ------------------------------------------------------------------------------
\subsection{Smoothing Window Optimization}
\label{subsec:window_optimization}
% ------------------------------------------------------------------------------

A 5-point moving average (15~mm window at 3~mm resolution) is applied to reduce measurement noise while preserving stratigraphic features.

\subsubsection{Window Size Comparison}

Figure~\ref{fig:window_optimization} compares window sizes of 1, 3, 5, 7, and 11 points on a representative section (TAM-5AB-6-7-B-RUN2):

\begin{itemize}
    \item \textbf{Window = 1:} Raw data; high noise obscures trends
    \item \textbf{Window = 3:} Reduced noise but still irregular (9~mm effective resolution)
    \item \textbf{Window = 5:} Optimal balance---clear trends with preserved sharp transitions (15~mm resolution)
    \item \textbf{Window = 7:} Over-smoothed; minor features lost (21~mm resolution)
    \item \textbf{Window = 11:} Excessive smoothing; stratigraphic detail obscured (33~mm resolution)
\end{itemize}

\begin{figure}[htbp]
\centering
\includegraphics[width=\textwidth]{../figures/filter_validation/fig_V5_window_size_optimization.png}
\caption{Window size optimization for moving average filter. Gray points = raw data; blue line = smoothed profile. Window = 5 (15~mm) provides optimal balance between noise reduction and feature preservation.}
\label{fig:window_optimization}
\end{figure}

% ------------------------------------------------------------------------------
\subsection{Alignment Verification}
\label{subsec:alignment_verification}
% ------------------------------------------------------------------------------

Figure~\ref{fig:alignment_verification} demonstrates that XRF data, optical images, and facies interpretations are correctly aligned stratigraphically:

\begin{itemize}
    \item \textbf{Visual correspondence:} Shell-rich layers (high Ca/Ti) correspond to visible shell concentrations in optical images
    \item \textbf{Transition coherence:} Facies boundaries align with lithological changes
    \item \textbf{Gap handling:} Inter-section gaps are correctly represented as gray bands
\end{itemize}

\begin{figure}[htbp]
\centering
\includegraphics[width=0.6\textwidth]{../figures/filter_validation/fig_V7_alignment_verification.png}
\caption{Stratigraphic alignment verification for GROUP3. Overlaid Ca/Ti (blue) and Fe/Mn (purple) profiles demonstrate coherent stratigraphic trends. Dotted horizontal lines mark key stratigraphic features that can be cross-referenced with optical images.}
\label{fig:alignment_verification}
\end{figure}

% ------------------------------------------------------------------------------
\subsection{Summary: Filter Validation}
\label{subsec:filter_summary}
% ------------------------------------------------------------------------------

Table~\ref{tab:filter_summary} summarizes all filter cutoffs with their spectral evidence.

\begin{table}[htbp]
\centering
\caption{Summary of validated filter parameters}
\label{tab:filter_summary}
\begin{tabular}{llll}
\toprule
Parameter & Cutoff & Evidence Type & Data Retention \\
\midrule
\multicolumn{4}{l}{\textit{QC Filters}} \\
MSE & $\leq$ 10 & Spectral fit quality & 99.4\% \\
CPS & $\geq$ 20,000 & Signal strength & 98.1\% \\
Surface & $<$ 8~mm & Detector distance & 96.3\% \\
\midrule
\multicolumn{4}{l}{\textit{Exclusion Zones}} \\
Foam/gaps & Manual + Inc/Coh & Spectral fingerprint & 80.2\% \\
\midrule
\multicolumn{4}{l}{\textit{Facies Thresholds (Ca/Ti)}} \\
Clastic & $<$ 2 & Natural breaks & --- \\
Mixed & 2--5 & Distribution inflection & --- \\
Carbonate & 5--10 & Lithological coherence & --- \\
Shell-rich & $>$ 10 & Visual confirmation & --- \\
\midrule
\multicolumn{4}{l}{\textit{Redox Threshold (Fe/Mn)}} \\
Oxic/Reducing & 50 & Literature + site discrimination & --- \\
\midrule
\multicolumn{4}{l}{\textit{Smoothing}} \\
Window size & 5 points & Signal-noise optimization & --- \\
\bottomrule
\end{tabular}
\end{table}

All filter parameters are empirically validated and documented with spectral evidence. The processing pipeline achieves 80.2\% data retention while ensuring measurement quality and meaningful geochemical interpretation.


% ==============================================================================
\section{Conclusions}
% ==============================================================================

This detailed stratigraphic documentation of XRF core scanner data from the Pebas Formation provides:

\begin{enumerate}
    \item \textbf{Comprehensive inventory:} 27 core sections from 7 groups, totaling 1,979 calibrated XRF measurements across 7.97~m of composite stratigraphy.

    \item \textbf{Validated proxy methodology:} Empirical redundancy analysis demonstrates that Ca/Ti is statistically redundant with raw Ca ($r$ = 0.89) in this dataset, while \FeMn{} ($r$ = 0.27) and \ZrRb{} ($r$ = 0.60) provide genuinely independent information. We recommend a streamlined four-proxy suite (Ca, Ti, \FeMn{}, \ZrRb{}) for Pebas Formation paleoenvironmental reconstruction.

    \item \textbf{Geochemical characterization:} Section-by-section documentation using the validated proxy suite, with facies classification based on Ca thresholds supported by the high variance (CV = 81\%) of carbonate signal.

    \item \textbf{Integrated visualization:} Optical core images aligned with geochemical profiles for each core group, with foam/gap sections identified and masked.

    \item \textbf{Site-specific signatures:} TAM records more reducing conditions (median \FeMn{} = 93.4) and higher carbonate content than SC (median \FeMn{} = 50.7), suggesting deeper or more restricted depositional settings at Tamshiyacu.

    \item \textbf{Reference data:} A detailed geochemical reference framework for ongoing paleontological, sedimentological, and isotopic studies of these cores.

    \item \textbf{CT integration framework:} Complementary CT scanning data (80 slices across two series) is available for integrated density-geochemistry analysis, with DICOM metadata enabling precise depth alignment with XRF measurements.
\end{enumerate}

% ==============================================================================
\section*{Data Availability}
% ==============================================================================

All XRF data, CT DICOM metadata, R scripts for data processing and visualization, and the Shiny application for interactive figure generation are available at: [repository URL]. CT DICOM image data is available upon request.

% ==============================================================================
% References
% ==============================================================================

\bibliographystyle{apalike}
\bibliography{references}

\end{document}
