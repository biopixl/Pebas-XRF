% ==============================================================================
% Pebas Formation XRF Core Scanner Stratigraphy
% Detailed Geochemical Documentation of Tamshiyacu and Santa Corina Cores
% ==============================================================================
\documentclass[11pt,a4paper]{article}

% Packages
\usepackage[utf8]{inputenc}
\usepackage[T1]{fontenc}
\usepackage{graphicx}
\usepackage{booktabs}
\usepackage{longtable}
\usepackage{siunitx}
\usepackage[margin=2.5cm]{geometry}
\usepackage{natbib}
\usepackage{hyperref}
\usepackage{xcolor}
\usepackage{lineno}
\usepackage{setspace}
\usepackage{rotating}
\usepackage{pdflscape}
\usepackage{multirow}
\usepackage{caption}
\usepackage{subcaption}

% Line numbers for review
\linenumbers
\onehalfspacing

% Custom commands
\newcommand{\ratio}[2]{#1/#2}
\newcommand{\CaTi}{Ca/Ti}
\newcommand{\FeMn}{Fe/Mn}
\newcommand{\KTi}{K/Ti}
\newcommand{\ZrRb}{Zr/Rb}

% Title
\title{Itrax XRF Core Scanner Stratigraphy of the Pebas Formation, Western Amazonia:\\
Detailed Geochemical Documentation of Tamshiyacu and Santa Corina Sediment Cores}

\author{Isaac Ayesu\textsuperscript{1}}

\date{\today}

\begin{document}

\maketitle

\begin{abstract}
This report presents a comprehensive stratigraphic and geochemical documentation of sediment cores from the Miocene Pebas Formation at two localities in the Peruvian Amazon: Tamshiyacu (TAM) and Santa Corina (SC). Seven core groups comprising 27 sections were analyzed using Itrax XRF core scanning at 3~mm resolution, yielding 1,979 calibrated measurements after quality control. We document the stratigraphic position, geochemical characteristics, and paleoenvironmental interpretation for each core section using established element ratio proxies (\CaTi{} for carbonate content, \FeMn{} for redox conditions, \KTi{} for weathering intensity, \ZrRb{} for grain size). The data reveal systematic differences between sites and depth-dependent variations in depositional environment. This detailed core-by-core documentation provides a reference framework for ongoing paleontological and sedimentological studies of the Pebas mega-wetland system.
\end{abstract}

\tableofcontents
\newpage

% ==============================================================================
\section{Introduction}
% ==============================================================================

\subsection{Geological Setting}

The Pebas Formation (middle to late Miocene, ca.\ 23--10 Ma) represents one of the largest freshwater to marginally brackish lake-wetland systems in Earth history, covering an estimated 1 million km$^2$ of western Amazonia \citep{Wesselingh2002, Hoorn2010}. The formation is characterized by cyclic sequences of lignites, mudstones, and shell beds deposited in shallow lacustrine, fluvial, and marginal marine environments associated with the Pebas mega-wetland system.

\subsection{Study Localities}

Two outcrop localities in the Peruvian Amazon were selected for detailed XRF core scanning:

\begin{description}
    \item[Tamshiyacu (TAM):] Located along the Amazon River approximately 30~km downstream of Iquitos (coordinates: 4.0017\textdegree S, 73.1567\textdegree W). Three core groups (GROUP1--3) spanning approximately 3.9~m of composite stratigraphy.

    \item[Santa Corina (SC):] Located on the Itaya River south of Iquitos (coordinates: 4.1833\textdegree S, 73.4333\textdegree W). Four core groups (GROUP4--7) spanning approximately 4.0~m of composite stratigraphy.
\end{description}

\subsection{Objectives}

This report provides:
\begin{enumerate}
    \item Complete inventory of all analyzed core sections with stratigraphic positions
    \item Section-by-section geochemical characterization using calibrated XRF data
    \item Integrated optical core images with geochemical profiles
    \item Facies classification based on element ratio thresholds
    \item Paleoenvironmental interpretation for each core group
\end{enumerate}

% ==============================================================================
\section{Materials and Methods}
% ==============================================================================

\subsection{Core Collection and Sampling}

Sediment cores were collected using a manual percussion coring system with 50~mm diameter aluminum core tubes. Cores were split longitudinally, described, and photographed before XRF analysis. Core sections were designated with site prefix (TAM or SC), core number, and section letter (A, B, C, etc.). Some sections required re-scanning (designated ``RUN2'').

\subsection{Itrax XRF Core Scanning}

All cores were analyzed using an Itrax XRF core scanner (Cox Analytical Systems, Sweden) at [Institution]. Scanning parameters are summarized in Table~\ref{tab:scan_params}.

\begin{table}[h]
\centering
\caption{Itrax XRF core scanner operating parameters}
\label{tab:scan_params}
\begin{tabular}{ll}
\toprule
Parameter & Value \\
\midrule
X-ray tube & Molybdenum (Mo) \\
Voltage & \SI{30}{kV} \\
Current & \SI{55}{mA} \\
Exposure time & \SI{10}{s} per measurement \\
Step size & \SI{3}{mm} \\
Optical image resolution & \SI{0.2}{mm} \\
Detector & Silicon drift detector (SDD) \\
\bottomrule
\end{tabular}
\end{table}

\subsection{Data Processing and Calibration}

Raw XRF spectra were processed using the Q-spec software with post-processing calibration (Results.txt files). Element concentrations are reported as calibrated peak areas proportional to concentration. Quality control criteria included:

\begin{itemize}
    \item Mean Squared Error (MSE) of spectral fit: $\leq 10$
    \item Total counts per second (cps): $\geq 20,000$
    \item Sample surface distance: $< \SI{8}{mm}$
    \item Visual inspection of optical images to identify foam, gaps, and artifacts
\end{itemize}

\subsection{Element Ratio Proxies}

Following established protocols for XRF core scanner paleoenvironmental reconstruction \citep{Croudace2015, Rothwell2015}, the following element ratios were calculated:

\begin{table}[h]
\centering
\caption{Element ratio proxies and interpretation criteria}
\label{tab:proxies}
\begin{tabular}{llll}
\toprule
Ratio & Proxy & Interpretation & Threshold \\
\midrule
\CaTi{} & Carbonate content & $>10$: Shell-rich, $5$--$10$: Carbonate & --- \\
        &                   & $2$--$5$: Mixed, $<2$: Clastic & \\
\FeMn{} & Redox conditions & $>50$: Reducing (anoxic) & 50 \\
        &                  & $<50$: Oxidizing (oxic) & \\
\KTi{}  & Weathering intensity & Higher: less weathered & --- \\
\ZrRb{} & Grain size proxy & Higher: coarser sediment & --- \\
\bottomrule
\end{tabular}
\end{table}

\subsection{Facies Classification}

A four-fold geochemical facies classification was applied based on \CaTi{} ratios:

\begin{description}
    \item[Shell-rich (\CaTi{} $>$ 10):] Biogenic carbonate-dominated intervals, typically containing mollusk shell accumulations
    \item[Carbonate (\CaTi{} 5--10):] Carbonate-enriched mudstones with mixed biogenic and authigenic carbonate
    \item[Mixed (\CaTi{} 2--5):] Transitional facies with balanced carbonate and terrigenous input
    \item[Clastic (\CaTi{} $<$ 2):] Terrigenous-dominated siliciclastic mudstones
\end{description}

% ==============================================================================
\section{Core Inventory}
% ==============================================================================

\subsection{Summary Statistics}

A total of 27 core sections from 7 core groups were analyzed, yielding 1,979 calibrated XRF measurements after quality control (Table~\ref{tab:inventory}).

\begin{table}[h]
\centering
\caption{Summary of analyzed core material by locality}
\label{tab:inventory_summary}
\begin{tabular}{lrrrrr}
\toprule
Locality & Groups & Sections & Measurements & Depth (m) & Excluded \\
\midrule
Tamshiyacu (TAM) & 3 & 11 & 930 & 3.93 & 208 \\
Santa Corina (SC) & 4 & 16 & 1,049 & 4.04 & 184 \\
\midrule
\textbf{Total} & \textbf{7} & \textbf{27} & \textbf{1,979} & \textbf{7.97} & \textbf{392} \\
\bottomrule
\end{tabular}
\end{table}

\subsection{Complete Section Inventory}

\begin{longtable}{llrrrrr}
\caption{Complete inventory of XRF core sections} \label{tab:inventory} \\
\toprule
Group & Section & Start (mm) & End (mm) & Length & n & Excluded \\
\midrule
\endfirsthead
\multicolumn{7}{c}{{\tablename\ \thetable{} -- continued}} \\
\toprule
Group & Section & Start (mm) & End (mm) & Length & n & Excluded \\
\midrule
\endhead
\midrule
\multicolumn{7}{r}{{Continued on next page}} \\
\endfoot
\bottomrule
\endlastfoot
% TAM cores
\multicolumn{7}{l}{\textbf{Tamshiyacu (TAM)}} \\
GROUP1 & TAM-1-2-3B-A & 123 & 660 & 537 & 153 & 68 \\
GROUP1 & TAM-1-2-3B-B & 901 & 1213 & 312 & 87 & 0 \\
GROUP1 & TAM-1-2-3B-C & 1322 & 1406 & 84 & 25 & 25 \\
\midrule
GROUP2 & TAM-3A-4-5CDE-A & 25 & 361 & 336 & 95 & 50 \\
GROUP2 & TAM-3A-4-5CDE-B & 448 & 1012 & 564 & 162 & 0 \\
GROUP2 & TAM-3A-4-5CDE-RUN2-C & 1175 & 1214 & 39 & 14 & 0 \\
GROUP2 & TAM-3A-4-5CDE-RUN2-D & 1312 & 1315 & 3 & 2 & 2 \\
GROUP2 & TAM-3A-4-5CDE-RUN2-E & 1384 & 1435 & 51 & 16 & 0 \\
\midrule
GROUP3 & TAM-5AB-6-7-A & 38 & 152 & 114 & 28 & 3 \\
GROUP3 & TAM-5AB-6-7-B-RUN2 & 210 & 654 & 444 & 111 & 13 \\
GROUP3 & TAM-5AB-6-7-C & 698 & 1170 & 472 & 237 & 57 \\
\midrule
% SC cores
\multicolumn{7}{l}{\textbf{Santa Corina (SC)}} \\
GROUP4 & SC-1ABC-2-3C-A-RUN1 & 34 & 364 & 330 & 92 & 5 \\
GROUP4 & SC-1ABC-2-3C-RUN2-B & 519 & 753 & 234 & 46 & 7 \\
GROUP4 & SC-1ABC-2-3C-RUN2-C & 938 & 1223 & 285 & 83 & 30 \\
\midrule
GROUP5 & SC-3AB-4ABCD-A & 37 & 253 & 216 & 61 & 8 \\
GROUP5 & SC-3AB-4ABCD-B & 327 & 456 & 129 & 22 & 2 \\
GROUP5 & SC-3AB-4ABCD-C & 538 & 661 & 123 & 38 & 0 \\
GROUP5 & SC-3AB-4ABCD-D & 739 & 885 & 146 & 74 & 0 \\
GROUP5 & SC-3AB-4ABCD-RUN2-D & 739 & 885 & 146 & 74 & 12 \\
GROUP5 & SC-3AB-4ABCD-RUN2-E & 919 & 1143 & 224 & 106 & 34 \\
GROUP5 & SC-3AB-4ABCD-RUN2-F & 1172 & 1374 & 202 & 96 & 0 \\
\midrule
GROUP6 & SC-5-6-7ABC-A & 24 & 210 & 186 & 43 & 15 \\
GROUP6 & SC-5-6-7ABC-B & 284 & 692 & 408 & 173 & 37 \\
GROUP6 & SC-5-6-7ABC-C & 776 & 875 & 99 & 28 & 0 \\
GROUP6 & SC-5-6-7ABC-D & 910 & 1015 & 105 & 22 & 11 \\
GROUP6 & SC-5-6-7ABC-E & 1053 & 1143 & 90 & 18 & 0 \\
\midrule
GROUP7 & SC8-A & 177 & 423 & 246 & 73 & 0 \\
\end{longtable}

% ==============================================================================
\section{Tamshiyacu (TAM) Core Stratigraphy}
% ==============================================================================

The Tamshiyacu locality comprises three core groups arranged in stratigraphic order from youngest (GROUP1, top) to oldest (GROUP3, bottom), with a total composite thickness of approximately 3.9~m.

\subsection{GROUP1: TAM-1-2-3B (Upper Section)}

\subsubsection{Core Description}

GROUP1 represents the uppermost stratigraphic interval at Tamshiyacu, comprising three sections (A, B, C) spanning positions 123--1406~mm (128.3~cm composite thickness). The core consists predominantly of gray to olive-gray mudstones with variable carbonate content and scattered shell fragments.

\subsubsection{Geochemical Characteristics}

\begin{table}[h]
\centering
\caption{GROUP1 (TAM) geochemical summary by section}
\label{tab:group1_geochem}
\begin{tabular}{lrrrrrrr}
\toprule
Section & n & \CaTi{} & $\sigma$ & \FeMn{} & \% Reducing & \% Shell-rich \\
\midrule
TAM-1-2-3B-A & 153 & 4.63 & 5.25 & 62.8 & 68.0 & 13.1 \\
TAM-1-2-3B-B & 87 & 4.17 & 7.32 & 83.4 & 77.0 & 12.6 \\
TAM-1-2-3B-C & 25 & 8.66 & 2.89 & 93.1 & 84.0 & 32.0 \\
\midrule
\textbf{GROUP1 Total} & 265 & 4.89 & 5.89 & 72.4 & 72.5 & 14.7 \\
\bottomrule
\end{tabular}
\end{table}

\begin{figure}[h]
\centering
\includegraphics[width=\textwidth]{figures/integrated_GROUP1.png}
\caption{GROUP1 (TAM-1-2-3B) integrated stratigraphy showing optical core image, geochemical facies classification, and proxy profiles (\CaTi{}, \FeMn{}). Dashed vertical lines indicate facies thresholds (\CaTi{} = 2, 5, 10; \FeMn{} = 50). Gray bands in core image indicate masked foam/gap zones.}
\label{fig:group1}
\end{figure}

\subsubsection{Interpretation}

GROUP1 records variable carbonate accumulation with intervals of shell-rich deposition (Section C, \CaTi{} mean = 8.66). Persistently elevated \FeMn{} ratios (62--93) indicate reducing bottom water conditions throughout. The upward decrease in \CaTi{} from Section C to A may reflect increasing terrigenous dilution or decreased carbonate production in the upper part of the sequence.

% --- GROUP2 ---
\subsection{GROUP2: TAM-3A-4-5CDE (Middle Section)}

\subsubsection{Core Description}

GROUP2 represents the middle stratigraphic interval at Tamshiyacu, comprising five sections (A, B, RUN2-C, RUN2-D, RUN2-E) spanning positions 25--1435~mm (141.0~cm composite thickness). This interval shows the highest carbonate content of any TAM core group.

\subsubsection{Geochemical Characteristics}

\begin{table}[h]
\centering
\caption{GROUP2 (TAM) geochemical summary by section}
\label{tab:group2_geochem}
\begin{tabular}{lrrrrrrr}
\toprule
Section & n & \CaTi{} & $\sigma$ & \FeMn{} & \% Reducing & \% Shell-rich \\
\midrule
TAM-3A-4-5CDE-A & 95 & 7.63 & 1.64 & 78.2 & 85.3 & 17.9 \\
TAM-3A-4-5CDE-B & 162 & 8.19 & 3.41 & 85.4 & 87.0 & 29.0 \\
TAM-3A-4-5CDE-RUN2-C & 14 & 4.68 & 0.75 & 72.1 & 71.4 & 0.0 \\
TAM-3A-4-5CDE-RUN2-D & 2 & 5.75 & 0.17 & 115.0 & 100.0 & 0.0 \\
TAM-3A-4-5CDE-RUN2-E & 16 & 5.27 & 0.26 & 69.8 & 75.0 & 0.0 \\
\midrule
\textbf{GROUP2 Total} & 289 & 7.69 & 2.91 & 81.3 & 84.4 & 22.5 \\
\bottomrule
\end{tabular}
\end{table}

\begin{figure}[h]
\centering
\includegraphics[width=\textwidth]{figures/integrated_GROUP2.png}
\caption{GROUP2 (TAM-3A-4-5CDE) integrated stratigraphy. This interval shows consistently elevated \CaTi{} values (mean = 7.69) indicating carbonate-rich deposition with shell-rich intervals in Sections A and B.}
\label{fig:group2}
\end{figure}

\subsubsection{Interpretation}

GROUP2 represents a carbonate-rich interval with mean \CaTi{} = 7.69, the highest among TAM groups. Sections A and B contain abundant shell-rich facies (18--29\% of measurements with \CaTi{} $>$ 10), suggesting periods of enhanced mollusk productivity. Strongly reducing conditions (\FeMn{} mean = 81.3) indicate persistent bottom water anoxia. The shorter RUN2 sections show lower carbonate content, possibly representing clay-rich interbeds.

% --- GROUP3 ---
\subsection{GROUP3: TAM-5AB-6-7 (Lower Section)}

\subsubsection{Core Description}

GROUP3 represents the lowermost stratigraphic interval at Tamshiyacu, comprising three sections (A, B-RUN2, C) spanning positions 38--1170~mm (113.2~cm composite thickness). This interval shows the most pronounced geochemical variability of any core group.

\subsubsection{Geochemical Characteristics}

\begin{table}[h]
\centering
\caption{GROUP3 (TAM) geochemical summary by section}
\label{tab:group3_geochem}
\begin{tabular}{lrrrrrrr}
\toprule
Section & n & \CaTi{} & $\sigma$ & \FeMn{} & \% Reducing & \% Shell-rich \\
\midrule
TAM-5AB-6-7-A & 28 & 3.22 & 1.68 & 63.5 & 64.3 & 3.6 \\
TAM-5AB-6-7-B-RUN2 & 111 & 10.20 & 5.90 & 95.3 & 88.3 & 47.7 \\
TAM-5AB-6-7-C & 237 & 1.75 & 0.44 & 67.2 & 64.1 & 0.0 \\
\midrule
\textbf{GROUP3 Total} & 376 & 4.41 & 4.87 & 76.0 & 72.1 & 15.4 \\
\bottomrule
\end{tabular}
\end{table}

\begin{figure}[h]
\centering
\includegraphics[width=\textwidth]{figures/integrated_GROUP3.png}
\caption{GROUP3 (TAM-5AB-6-7) integrated stratigraphy. Note the dramatic contrast between Section B-RUN2 (shell-rich, \CaTi{} mean = 10.20) and Section C (clastic, \CaTi{} mean = 1.75), representing a major facies transition.}
\label{fig:group3}
\end{figure}

\subsubsection{Interpretation}

GROUP3 records the most dramatic geochemical transition in the TAM sequence. Section B-RUN2 represents a prominent shell-rich interval (\CaTi{} mean = 10.20, 48\% shell-rich facies) with strongly reducing conditions (\FeMn{} = 95.3). In contrast, Section C shows clastic-dominated deposition (\CaTi{} mean = 1.75, 0\% shell-rich) with lower \FeMn{} values. This transition likely reflects a shift from carbonate-producing lacustrine conditions to increased terrigenous input, possibly related to changes in accommodation space or sediment supply.

% ==============================================================================
\section{Santa Corina (SC) Core Stratigraphy}
% ==============================================================================

The Santa Corina locality comprises four core groups arranged in stratigraphic order from youngest (GROUP4, top) to oldest (GROUP7, bottom), with a total composite thickness of approximately 4.0~m.

\subsection{GROUP4: SC-1ABC-2-3C (Upper Section)}

\subsubsection{Core Description}

GROUP4 represents the uppermost stratigraphic interval at Santa Corina, comprising three sections (A-RUN1, RUN2-B, RUN2-C) spanning positions 34--1223~mm (118.9~cm composite thickness). The core consists of gray mudstones with lower carbonate content than equivalent TAM intervals.

\subsubsection{Geochemical Characteristics}

\begin{table}[h]
\centering
\caption{GROUP4 (SC) geochemical summary by section}
\label{tab:group4_geochem}
\begin{tabular}{lrrrrrrr}
\toprule
Section & n & \CaTi{} & $\sigma$ & \FeMn{} & \% Reducing & \% Shell-rich \\
\midrule
SC-1ABC-2-3C-A-RUN1 & 92 & 1.00 & 0.38 & 38.4 & 34.8 & 0.0 \\
SC-1ABC-2-3C-RUN2-B & 46 & 3.59 & 1.00 & 71.2 & 78.3 & 0.0 \\
SC-1ABC-2-3C-RUN2-C & 83 & 4.09 & 1.63 & 68.4 & 71.1 & 2.4 \\
\midrule
\textbf{GROUP4 Total} & 221 & 2.70 & 1.83 & 56.4 & 57.0 & 0.9 \\
\bottomrule
\end{tabular}
\end{table}

\begin{figure}[h]
\centering
\includegraphics[width=\textwidth]{figures/integrated_GROUP4.png}
\caption{GROUP4 (SC-1ABC-2-3C) integrated stratigraphy. Section A shows distinctly lower \CaTi{} and \FeMn{} values compared to Sections B and C, suggesting more oxic and terrigenous-dominated conditions.}
\label{fig:group4}
\end{figure}

\subsubsection{Interpretation}

GROUP4 shows significant intra-group variability. Section A records predominantly clastic deposition (\CaTi{} = 1.00) under relatively oxic conditions (\FeMn{} = 38.4, only 35\% reducing). Sections B and C show increased carbonate content and more reducing conditions. This upward transition from oxic-clastic to reducing-carbonate facies may reflect deepening water conditions or reduced sediment supply.

% --- GROUP5 ---
\subsection{GROUP5: SC-3AB-4ABCD (Upper-Middle Section)}

\subsubsection{Core Description}

GROUP5 is the most extensively sampled core group, comprising seven sections spanning positions 37--1374~mm (133.7~cm composite thickness). Multiple re-scans (RUN2) were required due to initial scanning issues.

\subsubsection{Geochemical Characteristics}

\begin{table}[h]
\centering
\caption{GROUP5 (SC) geochemical summary by section}
\label{tab:group5_geochem}
\begin{tabular}{lrrrrrrr}
\toprule
Section & n & \CaTi{} & $\sigma$ & \FeMn{} & \% Reducing & \% Shell-rich \\
\midrule
SC-3AB-4ABCD-A & 61 & 2.57 & 0.81 & 41.8 & 41.0 & 0.0 \\
SC-3AB-4ABCD-B & 22 & 2.36 & 0.81 & 43.1 & 45.5 & 0.0 \\
SC-3AB-4ABCD-C & 38 & 4.90 & 2.75 & 55.6 & 55.3 & 7.9 \\
SC-3AB-4ABCD-D & 74 & 4.09 & 1.55 & 61.2 & 63.5 & 2.7 \\
SC-3AB-4ABCD-RUN2-D & 74 & 4.11 & 1.57 & 63.1 & 66.2 & 2.7 \\
SC-3AB-4ABCD-RUN2-E & 106 & 6.78 & 2.41 & 67.8 & 72.6 & 10.4 \\
SC-3AB-4ABCD-RUN2-F & 96 & 6.62 & 2.40 & 57.2 & 58.3 & 11.5 \\
\midrule
\textbf{GROUP5 Total} & 471 & 4.84 & 2.51 & 57.4 & 58.8 & 5.7 \\
\bottomrule
\end{tabular}
\end{table}

\begin{figure}[h]
\centering
\includegraphics[width=\textwidth]{figures/integrated_GROUP5.png}
\caption{GROUP5 (SC-3AB-4ABCD) integrated stratigraphy. The sequence shows an upward-increasing trend in \CaTi{} values from Sections A--B (clastic) to Sections E--F (carbonate-enriched), accompanied by increasingly reducing conditions.}
\label{fig:group5}
\end{figure}

\subsubsection{Interpretation}

GROUP5 records a systematic upward increase in carbonate content and reducing conditions. The basal sections (A, B) are clastic-dominated (\CaTi{} $<$ 3) with borderline oxic/reducing conditions (\FeMn{} $\approx$ 42). Upper sections (E, F) show elevated carbonate (\CaTi{} $>$ 6) and shell-rich intervals (10--12\% of measurements). This progression suggests gradual deepening and/or reduced terrigenous input through time.

% --- GROUP6 ---
\subsection{GROUP6: SC-5-6-7ABC (Lower-Middle Section)}

\subsubsection{Core Description}

GROUP6 comprises five sections (A--E) spanning positions 24--1143~mm (111.9~cm composite thickness). This interval shows intermediate geochemical characteristics between the overlying GROUP5 and underlying GROUP7.

\subsubsection{Geochemical Characteristics}

\begin{table}[h]
\centering
\caption{GROUP6 (SC) geochemical summary by section}
\label{tab:group6_geochem}
\begin{tabular}{lrrrrrrr}
\toprule
Section & n & \CaTi{} & $\sigma$ & \FeMn{} & \% Reducing & \% Shell-rich \\
\midrule
SC-5-6-7ABC-A & 43 & 2.28 & 0.85 & 45.3 & 44.2 & 0.0 \\
SC-5-6-7ABC-B & 173 & 3.00 & 3.15 & 54.1 & 54.9 & 5.8 \\
SC-5-6-7ABC-C & 28 & 1.74 & 1.07 & 48.5 & 46.4 & 0.0 \\
SC-5-6-7ABC-D & 22 & 5.08 & 8.76 & 61.8 & 59.1 & 18.2 \\
SC-5-6-7ABC-E & 18 & 3.44 & 2.01 & 56.2 & 55.6 & 5.6 \\
\midrule
\textbf{GROUP6 Total} & 284 & 3.04 & 3.45 & 52.6 & 52.5 & 5.3 \\
\bottomrule
\end{tabular}
\end{table}

\begin{figure}[h]
\centering
\includegraphics[width=\textwidth]{figures/integrated_GROUP6.png}
\caption{GROUP6 (SC-5-6-7ABC) integrated stratigraphy. Geochemical variability is pronounced, with Section D showing a distinct shell-rich interval (\CaTi{} mean = 5.08, 18\% shell-rich).}
\label{fig:group6}
\end{figure}

\subsubsection{Interpretation}

GROUP6 shows heterogeneous geochemical signatures with alternating clastic and carbonate-enriched intervals. Section D stands out with elevated \CaTi{} (mean = 5.08) and the highest proportion of shell-rich facies (18\%) in the SC sequence. The high standard deviation in Section D (\CaTi{} $\sigma$ = 8.76) indicates punctuated shell accumulation events rather than sustained carbonate production.

% --- GROUP7 ---
\subsection{GROUP7: SC8-A (Lower Section)}

\subsubsection{Core Description}

GROUP7 represents the lowermost stratigraphic interval at Santa Corina, comprising a single section (SC8-A) spanning positions 177--423~mm (24.6~cm thickness). This is the shortest core group but provides important basal context for the SC sequence.

\subsubsection{Geochemical Characteristics}

\begin{table}[h]
\centering
\caption{GROUP7 (SC) geochemical summary}
\label{tab:group7_geochem}
\begin{tabular}{lrrrrrrr}
\toprule
Section & n & \CaTi{} & $\sigma$ & \FeMn{} & \% Reducing & \% Shell-rich \\
\midrule
SC8-A & 73 & 2.41 & 5.68 & 68.3 & 71.2 & 8.2 \\
\bottomrule
\end{tabular}
\end{table}

\begin{figure}[h]
\centering
\includegraphics[width=\textwidth]{figures/integrated_GROUP7.png}
\caption{GROUP7 (SC8-A) integrated stratigraphy. Despite low mean \CaTi{} (2.41), this section shows scattered shell-rich intervals (8\% of measurements) under reducing conditions.}
\label{fig:group7}
\end{figure}

\subsubsection{Interpretation}

GROUP7 is predominantly clastic (\CaTi{} mean = 2.41) but contains discrete shell-rich intervals, as indicated by the high standard deviation ($\sigma$ = 5.68). The elevated \FeMn{} ratio (68.3) indicates reducing bottom water conditions. This interval may represent episodic mollusk colonization events within an otherwise terrigenous-dominated setting.

% ==============================================================================
\section{Composite Stratigraphic Profiles}
% ==============================================================================

\subsection{Tamshiyacu Composite Section}

Figure~\ref{fig:strat_tam} presents the complete composite stratigraphic profile for Tamshiyacu, with all three core groups stacked in correct stratigraphic order.

\begin{figure}[h]
\centering
\includegraphics[width=\textwidth]{figures/stacked_TAM.png}
\caption{Composite stratigraphic column for Tamshiyacu (TAM) cores showing stacked groups (GROUP1 = youngest/top, GROUP3 = oldest/bottom). Total composite thickness $\approx$ 3.9~m. Note the persistent \FeMn{} values above the oxic/anoxic threshold (50) indicating reducing conditions throughout, with the most intense reduction in GROUP2.}
\label{fig:strat_tam}
\end{figure}

\subsection{Santa Corina Composite Section}

Figure~\ref{fig:strat_sc} presents the complete composite stratigraphic profile for Santa Corina, with all four core groups stacked in correct stratigraphic order.

\begin{figure}[h]
\centering
\includegraphics[width=\textwidth]{figures/stacked_SC.png}
\caption{Composite stratigraphic column for Santa Corina (SC) cores showing stacked groups (GROUP4 = youngest/top, GROUP7 = oldest/bottom). Total composite thickness $\approx$ 4.0~m. Santa Corina shows more variable redox conditions than TAM, with some intervals near or below the oxic/anoxic threshold.}
\label{fig:strat_sc}
\end{figure}

% ==============================================================================
\section{Facies Distribution and Paleoenvironmental Synthesis}
% ==============================================================================

\subsection{Facies Summary}

Table~\ref{tab:facies_summary} summarizes the distribution of geochemical facies across all core groups.

\begin{table}[h]
\centering
\caption{Facies distribution by core group (percentage of measurements)}
\label{tab:facies_summary}
\begin{tabular}{lrrrr}
\toprule
Group & Clastic & Mixed & Carbonate & Shell-rich \\
      & (\CaTi{} $<$ 2) & (2--5) & (5--10) & ($>$ 10) \\
\midrule
\multicolumn{5}{l}{\textbf{Tamshiyacu (TAM)}} \\
GROUP1 & 18.1 & 47.2 & 20.0 & 14.7 \\
GROUP2 & 1.4 & 26.0 & 50.2 & 22.5 \\
GROUP3 & 36.2 & 32.7 & 15.7 & 15.4 \\
\midrule
\multicolumn{5}{l}{\textbf{Santa Corina (SC)}} \\
GROUP4 & 38.5 & 50.2 & 10.4 & 0.9 \\
GROUP5 & 13.6 & 50.7 & 30.0 & 5.7 \\
GROUP6 & 35.9 & 43.0 & 15.8 & 5.3 \\
GROUP7 & 50.7 & 28.8 & 12.3 & 8.2 \\
\bottomrule
\end{tabular}
\end{table}

\subsection{Site Comparison}

Key differences between Tamshiyacu and Santa Corina:

\begin{enumerate}
    \item \textbf{Carbonate content:} TAM shows significantly higher carbonate content (mean \CaTi{} = 5.5) compared to SC (mean \CaTi{} = 3.5), with 17\% shell-rich facies at TAM versus only 4\% at SC.

    \item \textbf{Redox conditions:} TAM records more consistently reducing conditions (mean \FeMn{} = 76) compared to SC (mean \FeMn{} = 56), suggesting deeper water or greater organic matter accumulation at the Tamshiyacu locality.

    \item \textbf{Stratigraphic trends:} Both localities show significant vertical variability in \CaTi{}, but TAM GROUP2 represents the most carbonate-rich interval in either sequence. SC GROUP4 (uppermost) is the most terrigenous-dominated interval.
\end{enumerate}

\subsection{Paleoenvironmental Interpretation}

The geochemical data support the following paleoenvironmental reconstruction:

\begin{description}
    \item[Depositional setting:] Both localities record deposition within the Pebas mega-wetland system, characterized by fine-grained sedimentation (low \ZrRb{}), variable carbonate accumulation, and predominantly reducing bottom water conditions.

    \item[Spatial heterogeneity:] Higher carbonate content and more reducing conditions at TAM suggest deposition in a more distal, deeper, or more restricted portion of the lake system compared to SC.

    \item[Temporal variability:] Pronounced vertical variations in \CaTi{} and \FeMn{} at both sites indicate fluctuating environmental conditions, likely related to changes in water depth, terrigenous input, and/or biological productivity.

    \item[Shell-rich intervals:] Discrete shell-rich horizons (\CaTi{} $>$ 10) represent periods of enhanced mollusk productivity and/or reduced sediment dilution, providing targets for paleontological investigation.
\end{description}

% ==============================================================================
\section{Conclusions}
% ==============================================================================

This detailed stratigraphic documentation of XRF core scanner data from the Pebas Formation provides:

\begin{enumerate}
    \item \textbf{Comprehensive inventory:} 27 core sections from 7 groups, totaling 1,979 calibrated XRF measurements across 7.97~m of composite stratigraphy.

    \item \textbf{Geochemical characterization:} Section-by-section documentation of element ratios (\CaTi{}, \FeMn{}, \KTi{}, \ZrRb{}) with facies classification based on established thresholds.

    \item \textbf{Integrated visualization:} Optical core images aligned with geochemical profiles for each core group, with foam/gap sections identified and masked.

    \item \textbf{Paleoenvironmental framework:} Evidence for deposition within the Pebas mega-wetland under predominantly reducing conditions, with spatial differences between localities and significant temporal variability in carbonate production.

    \item \textbf{Reference data:} A detailed geochemical reference framework for ongoing paleontological, sedimentological, and isotopic studies of these cores.
\end{enumerate}

% ==============================================================================
\section*{Data Availability}
% ==============================================================================

All XRF data, R scripts for data processing and visualization, and the Shiny application for interactive figure generation are available at: [repository URL].

% ==============================================================================
% References
% ==============================================================================

\bibliographystyle{apalike}
\bibliography{references}

\end{document}
