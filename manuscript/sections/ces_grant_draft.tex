% ==============================================================================
% CES GRANT DRAFT - Center for Evolutionary Science Winter 2026
% One-page proposal for pXRF validation of fossil provenance
% ==============================================================================

\section*{CES Grant Draft: Non-Destructive Geochemical Fingerprinting of Fossil Megafauna}
\label{sec:ces_grant}

% ---------------------------------------------------------------------------
% PROPOSAL TEXT (~1 page)
% ---------------------------------------------------------------------------

\subsection*{Background and Significance}

The evolution of diet and body size is an important innovation in the history of life on Earth. A fossil record of megafauna from diverse ecosystems offers an archive of impressive strategies to investigate the fidelity of chemical signatures diagnostic to the site of collection. Museums and research institutions inherit legacy collections that can benefit from additional independent lines of evidence for supporting provenance without destructive sampling. X-Ray Fluorescence (XRF) is a reliable spectroscopy method in geology that can reveal the chemical composition of a fossil. However, fundamental questions about the depth of x-ray penetration, chemical/physical matrix effects, and post-mortem diagenesis need to be resolved before the assemblage and analysis of large XRF data frames. At Caltech, this project will leverage the instrument resources and active field research in the GPS division to investigate invaluable museum collections with referenced localities from California and Peru.

\subsection*{Preliminary Data}

Our pilot study of 48 Pebas Formation (Miocene, Peru) vertebrate fossils from the American Museum of Natural History (n=17) and Universidad Nacional Mayor de San Marcos, Lima (n=31) demonstrates the potential for pXRF-based provenance assessment. Key findings include:

\begin{itemize}
    \item \textbf{Inter-institutional signatures:} Specimens from the same geological formation but different museum collections exhibit systematic elemental differences (Ca: 234,561 vs. 188,761 ppm; Sr: 6,528 vs. 825 ppm), enabling discrimination of collection history.

    \item \textbf{Diagenetic classification:} Elemental ratios (Ca/P, Si/Ca) successfully classify preservation quality---AMNH specimens show better phosphate preservation (Ca/P = 3.1), while UNMSM material exhibits greater silicification (Si/Ca = 0.42).

    \item \textbf{Paleoenvironmental validation:} Fossil Fe concentrations ($\sim$32,000 ppm) and Sr/Ca ratios independently confirm the freshwater mega-wetland depositional environment reconstructed from sediment geochemistry (Itrax core scanning of TAM and SC localities).
\end{itemize}

\subsection*{Proposed Research}

This project will establish a validated pXRF protocol for non-destructive provenance fingerprinting of fossil vertebrates by:

\begin{enumerate}
    \item \textbf{Expanding the reference database:} Analyze Barstow Formation (Miocene, California) camelid specimens at AMNH (n$\approx$40) alongside Peru material to test geographic discrimination using the same taxonomic group across distinct depositional environments.

    \item \textbf{Quantifying analytical uncertainty:} Conduct replicate measurements and pressed-pellet standards to establish detection limits, precision, and accuracy for bone/tooth matrices.

    \item \textbf{Developing diagenetic correction factors:} Use paired bulk/microsampling to model the relationship between surface pXRF signals and internal bone chemistry, addressing the x-ray penetration depth question.
\end{enumerate}

\subsection*{Evolutionary Significance}

Megafaunal body size evolution in Miocene South America occurred during dramatic environmental transitions---from the Pebas mega-wetland system to the modern Amazon drainage. Non-destructive geochemical fingerprinting enables large-scale comparative studies of fossil collections without sacrificing material for isotopic analysis, preserving specimens for future research while extracting paleobiological information about diet, habitat, and taphonomic history. This methodology will support investigations into how Miocene megafauna (giant crocodilians, notoungulates, xenarthrans) adapted to changing freshwater ecosystems, directly addressing the \textit{history of life on Earth} and \textit{biotic consequences of climate change} priorities of the Center for Evolutionary Science.

\subsection*{Budget Summary}

\begin{tabular}{lr}
\toprule
Item & Cost \\
\midrule
Graduate student support (1 quarter) & \$18,000 \\
pXRF instrument time and consumables & \$8,000 \\
Museum travel (AMNH New York, 2 trips) & \$6,000 \\
Reference standards and sample preparation & \$4,000 \\
Conference presentation (SVP Annual Meeting) & \$4,000 \\
\midrule
\textbf{Total} & \textbf{\$40,000} \\
\bottomrule
\end{tabular}

% ==============================================================================
% END CES GRANT DRAFT
% ==============================================================================
